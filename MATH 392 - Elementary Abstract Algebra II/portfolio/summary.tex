%!TEX output_directory = summTemp
\documentclass[letterpaper, 12pt]{article}
	%%%%%%%%%%%%%%%%%%%%%%%%%%%%%%%%%%%%%%%%%%%%%%%%%%%%%%%%%%%%%%%%%%%%%%%%%%%%%%
	%%%%%%%%%%%%%%%%%%%%%%%%%%%% boilerplate packages %%%%%%%%%%%%%%%%%%%%%%%%%%%%
	\usepackage{amsmath,amssymb,amsthm}
	\usepackage[mathscr]{euscript}
	\usepackage{enumerate}
	\usepackage{graphicx}
	\usepackage{mathrsfs}
	\usepackage{color}
	% \usepackage{hyperref}
	\usepackage{verbatim}
	\usepackage[margin=1.25in]{geometry}

	%%%%%%%%%%%%%%%%%%%%%%%%%%%%%%%%%%%%%%%%%%%%%%%%%%%%%%%%%%%%%%%%%%%%%%%%%%%%%%
	%%%%%%%%%%%%%%%%%%%%%%%%%%%%% rename the abstract %%%%%%%%%%%%%%%%%%%%%%%%%%%%
	% \renewcommand{\abstractname}{Introduction}

	%%%%%%%%%%%%%%%%%%%%%%%%%%%%%%%%%%%%%%%%%%%%%%%%%%%%%%%%%%%%%%%%%%%%%%%%%%%%%%
	%%%%%%%%%%%%%%%%%%%%%%%%%%%%%%%%%%%%% sets %%%%%%%%%%%%%%%%%%%%%%%%%%%%%%%%%%%
		%% sets 
		\DeclareMathOperator{\N}{\mathbb{N}}
		\DeclareMathOperator{\Z}{\mathbb{Z}}
		\DeclareMathOperator{\Zp}{\mathbb{Z}^{+}}
		\DeclareMathOperator{\Q}{\mathbb{Q}}
		\DeclareMathOperator{\Qp}{\mathbb{Q}^{+}}
		\DeclareMathOperator{\Qc}{\mathbb{Q}^{c}}
		\DeclareMathOperator{\R}{\mathbb{R}}
		\DeclareMathOperator{\Rp}{\mathbb{R}^{+}}
		\DeclareMathOperator{\C}{\mathbb{C}}
		\DeclareMathOperator{\Cnon}{\mathbb{C}^{\times}}
		%% powerset of a set
		\DeclareMathOperator{\pset}{\mathcal{P}}
		%% set of continuous functions in a certain variable
		\DeclareMathOperator{\cont}{\mathscr{C}}
		%% set of functions in a certain variable
		\DeclareMathOperator{\func}{\mathscr{F}}
		
	%%%%%%%%%%%%%%%%%%%%%%%%%%%%%%%%%%%%%%%%%%%%%%%%%%%%%%%%%%%%%%%%%%%%%%%%%%%%%%
	%%%%%%%%%%%%%%%%%%%%%%%%%%%%%%%% linear algebra %%%%%%%%%%%%%%%%%%%%%%%%%%%%%%
		%% linear span
		\DeclareMathOperator{\Ell}{\mathscr{L}}
		%% bold vectors with arrows, and bold matrices
		\newcommand{\bmat}[1]{{\mathbf{#1}}}
		\newcommand{\bvec}[1]{{\vec{\mathbf{#1}}}}
		%% independent vectors/matrices
		\DeclareMathOperator{\ind}{\perp\!\!\!\perp}
		%% order
		\newcommand{\ord}[1]{\text{ord}(#1)}
		\renewcommand{\char}[1]{\text{char}(#1)}

	%%%%%%%%%%%%%%%%%%%%%%%%%%%%%%%%%%%%%%%%%%%%%%%%%%%%%%%%%%%%%%%%%%%%%%%%%%%%%%
	%%%%%%%%%%%%%%%%%%%%%%%%%%% probability & statistics %%%%%%%%%%%%%%%%%%%%%%%%%
		%% probability, expectation, variance, etc.
		\renewcommand{\Pr}{\mathbb{P}}
		\DeclareMathOperator{\E}{\mathbb{E}}
		\DeclareMathOperator{\var}{\rm Var}
		\DeclareMathOperator{\sd}{\rm SD}
		\DeclareMathOperator{\cov}{\rm Cov}
		\DeclareMathOperator{\SE}{\rm SE}
		\DeclareMathOperator{\ssreg}{{\rm SS}_{{\rm Reg}}}
		\DeclareMathOperator{\ssr}{{\rm SS}_{{\rm Res}}}
		\DeclareMathOperator{\sst}{{\rm SS}_{{\rm Tot}}}

	%%%%%%%%%%%%%%%%%%%%%%%%%%%%%%%%%%%%%%%%%%%%%%%%%%%%%%%%%%%%%%%%%%%%%%%%%%%%%%
	%%%%%%%%%%%%%%%%%%%%%%%%%%%%%%%% congruences %%%%%%%%%%%%%%%%%%%%%%%%%%%%%%%%%
		\renewcommand{\mod}[1]{\ (\mathrm{mod}\ #1)}

	%%%%%%%%%%%%%%%%%%%%%%%%%%%%%%%%%%%%%%%%%%%%%%%%%%%%%%%%%%%%%%%%%%%%%%%%%%%%%%
	%%%%%%%%%%%%%%%%%%%%%%%%%%%%%% bracket notation %%%%%%%%%%%%%%%%%%%%%%%%%%%%%%
		% I first used this for principal ideals, that is why the abbreviation is pid
		\newcommand{\pid}[1]{\langle #1 \rangle}

	%%%%%%%%%%%%%%%%%%%%%%%%%%%%%%%%%%%%%%%%%%%%%%%%%%%%%%%%%%%%%%%%%%%%%%%%%%%%%%
	%%%%%%%%%%%%%%%%%%%%%%%%%%%%%%% fatdot notation %%%%%%%%%%%%%%%%%%%%%%%%%%%%%%
		\makeatletter
			\newcommand*\fatdot{\mathpalette\fatdot@{.5}}
			\newcommand*\fatdot@[2]{\mathbin{\vcenter{\hbox{\scalebox{#2}{$\m@th#1\bullet$}}}}}
		\makeatother

	%%%%%%%%%%%%%%%%%%%%%%%%%%%%%%%%%%%%%%%%%%%%%%%%%%%%%%%%%%%%%%%%%%%%%%%%%%%%%%
	%%%%%%%%%%%%%%%%%%%%%%%%%%%%%% use pretty letters %%%%%%%%%%%%%%%%%%%%%%%%%%%%
		\DeclareMathOperator{\ep}{\varepsilon}
		\DeclareMathOperator{\ph}{\varphi}

	%%%%%%%%%%%%%%%%%%%%%%%%%%%%%%%%%%%%%%%%%%%%%%%%%%%%%%%%%%%%%%%%%%%%%%%%%%%%%%
	%%%%%%%%%%%%%%%%%%%%%%%%%%%%%%%%%%% theorems %%%%%%%%%%%%%%%%%%%%%%%%%%%%%%%%%
		\newtheorem{defn}{Definition}
		\newtheorem{thm}{Theorem}
		\newtheorem{cor}{Corollary}
\begin{document}
	\pagenumbering{gobble}
	\title{Summary Portfolio}
	\author{Alex Thies \\ {\lowercase{athies$@$uoregon.edu}}}

	\maketitle

	\tableofcontents

	\newpage
	
	\pagenumbering{arabic}
	\setcounter{section}{6}
	\section{Polynomials over a Ring}
	\label{sec:polynomials_over_a_ring}
		\subsection{Polynomials over a Ring}
		\label{sub:polynomials_over_a_ring}
			\begin{defn}
			Let $A$ be a commutative ring with unity. 
			For each nonnegative integer $n$ and elements $a_{0}, a_{1}, \dots , a_{n} \in A$ we can define a polynomial over $A$, $a(x)$, by: $$a(x) = a_{0} + a_{1}x + a_{2}x^{2} + \cdots + a_{n}x^{n} \ \ \ \ \ \rm or \it \ \ \ \ \ \sum\limits_{i=0}^{n}a_{i}x^{i}.$$
			The set of all polynomials over a ring $A$ is denoted $A[x]$.
			\end{defn}
			\setcounter{defn}{3}
			\begin{defn}
			Suppose $A$ is a commutative ring with unity and $a(x) \in A[x]$ with $a(x) = a_{0} + a_{1}x + \cdots + a_{n}x^{n}$ for some nonnegative integer $n$.
				\begin{enumerate}[(i)]
				\item The elements $a_{0},a_{1}, \dots , a_{n} \in A$ are the coefficients of $a(x)$.
				\item For each $0 \leq i \leq n$, $a_{i}x^{i}$ is called a term of $a(x)$.
				\item The largest nonnegative integer $n$ with $a_{n} \neq 0_{A}$ (if one exists) is the degree of $a(x)$, denoted $\deg(a(x)) = n$. 
				So for $k > n$ we know $a_{k} = 0_{A}$.
				\item If all coefficients of $a(x)$ are $0_{A}$ we say the degree of $a(x)$ is $-\infty$.
				\item For $n \geq 0$ if $\deg(a(x)) = n$ then $a_{n}$ is called the leading coefficient of $a(x)$.
				\end{enumerate}
			\end{defn}

			\begin{defn}
			Let $A$ be a commutative ring with unity. 
			For polynomials $a(x), b(x) \in A[x]$ we say $a(x) = b(x)$ if and only if they have the same degree and if the degree is equal to $n \geq 0$ then for every $i \leq n$, $a_{i} = b_{i}$.
			\end{defn} 

			\begin{defn}
			Let $A$ be a commutative ring with unity and let $a(x), b(x) \in A[x]$ as shown below. $$a(x) = \sum\limits_{i=0}^{n}a_{i}x^{i} \hspace{2cm} b(x) = \sum\limits_{i=0}^{m}b_{i}x^{i}$$
			We define the new polynomial $c(x) = a(x) + b(x)$ as follows where $k = \max\{ n,m \}$. $$c(x) = \sum\limits_{i=0}^{k}c_{i}x^{i} \hspace{5mm} and \hspace{5mm} c_{i}=a_{i}+b_{i}$$
			Remember, if $i > n$ or $i > m$ we assume $a = 0_{A}$ or $b = 0_{A}$, respectively.
			\end{defn}

			\setcounter{defn}{7}
			\begin{defn}
			Let $A$ be a commutative ring with unity and polynomials $a(x), b(x) \in A[x]$ as shown below. $$a(x) = \sum\limits_{i=0}^{n}a_{i}x^{i} \hspace{2cm} b(x) = \sum\limits_{i=0}^{m}b_{i}x^{i}$$
			Define the new polynomial $d(x) = a(x)b(x)$ as follows. $$d(x) = \sum\limits_{i=0}^{n+m}d_{i}x^{i} \hspace{5mm} where \hspace{5mm} d_{i} = \sum\limits_{j+t=i} a_{j} \cdot_{A} b_{t}$$
			Note: $0 \leq j \leq n$ and $0 \leq t \leq m$.
			\end{defn}

			\setcounter{thm}{10}
			\begin{thm}
			Let $A$ be a commutative ring with unity. 
			The operations of polynomial addition and polynomial multiplication from Definitions 7.6 and 7.8 are associative in $A[x]$.
			\end{thm}

			\setcounter{thm}{12}
			\begin{thm}
			Let $A$ be a commutative ring with unity. 
			In $A[x]$, polynomial addition and polynomial multiplication are both commutative.
			\end{thm}

			\begin{thm}
			Let $A$ be a commutative ring with unity. 
			Then the distributive laws hold in $A[x]$.
			\end{thm}

			\begin{thm}
			Let $A$ be a commutative ring with unity. 
			Then the set $A[x]$ of polynomials over $A$ is a commutative ring with unity.
			\end{thm}
		% subsection polynomials_over_a_ring (end)

		\subsection{Properties of Polynomial Rings}
		\label{sub:properties_of_polynomial_rings}
			\setcounter{thm}{16}
			\begin{thm}
			If $A$ is an integral domain then $A[x]$ is also an integral domain.
			\end{thm}

			\setcounter{thm}{19}
			\begin{thm}
			Let $A$ be an \textit{integral domain}, and nonzero $a(x), b(x) \in A[x]$. 
			If $\deg(a(x)) = n$ and $\deg(b(x)) = m$, then $\deg(a(x)b(x)) = n + m$.
			\end{thm}

			\setcounter{thm}{21}
			\begin{thm}
			If $A$ is a commutative ring with unity then $\char{A} = \char{A[x]}$.
			\end{thm}

			\setcounter{thm}{23}
			\begin{thm}[The Division Algorithm]
			Let $K$ be a field and $a(x),b(x) \in K[x]$. 
			If $b(x) \neq 0(x)$ then there exist unique polynomials $q(x),r(x) \in K[x]$, for which $a(x) = b(x)q(x) + r(x)$ and either $\deg(r(x)) < \deg(b(x))$ or $r(x) = 0(x)$.
			\end{thm}

			\setcounter{thm}{25}
			\begin{thm}
			Let $K$ be a field. 
			Then every ideal of $K[x]$ is a principal ideal.
			\end{thm}

			\begin{thm}
			Let $A$ be a commutative ring with unity. 
			Then the function $f : A \to A[x]$ defined by $f(a) = a + 0_{A}x$ is an injective ring homomorphism.
			\end{thm}

			\begin{thm}
			Let $A$ and $K$ be commutative rings with unity, and suppose that $f : A \to K$ is a ring homomorphism. 
			Then the function $\bar{f} : A[x] \to K[x]$ defined below is also a ring homomorphism. $$\bar{f}(a_{0} + a_{1}x + \dots + a_{n}x^{n}) = f(a_{0}) + f(a_{1})x + \dots + f(a_{n})x^{n}$$
			\end{thm}

			\setcounter{thm}{29}
			\begin{thm}
			Let $A, K$ be commutative rings with unity, and suppose that $f:A \to K$ is an isomorphism.
			Then the extension $\bar{f} : A[x] \to K[x]$ is also an isomorphism.
			\end{thm}
		% subsection properties_of_polynomial_rings (end)

		\subsection{Polynomial Functions and Roots}
		\label{sub:polynomial_functions_and_roots}
			\setcounter{defn}{32}
			\begin{defn}
			Let $A$ be a commutative ring with unity and $a(x) \in A[x]$ with $a(x) \neq 0(x)$. 
			If $c \in A$ and $\deg(a(x)) = n$, we define the element $a(c) \in A$ as follows: $$a(c) = a_{0} +_{A} (a_{1} \cdot_{A} c) +_{A} (a_{2} \cdot_{A} c^{2}) +_{A} \cdots +_{A} (a_{n} \cdot_{A} c^{n}).$$
			If $a(x) = 0(x)$ we say $a(c) = 0_{A}$ for all $c \in A$.
			\end{defn}

			\setcounter{thm}{34}
			\begin{thm}
			Let $A$ be an integral domain. 
			The substitution function $h_{c}: A[x] \to A$ defined by $h_{c}(a(x)) = a(c))$ is a ring homomorphism.
			\end{thm}

			\setcounter{defn}{36}
			\begin{defn}
			Let $A$ be a commutative ring with unity, $c \in A$, and $a(x) \in A[x] \ \ a(x) \neq 0(x)$.
			We say that $c$ is a root of the polynomial $a(x)$ exactly when $a(c) = 0_{A}$. 
			We do not say any element of $A$ is a root of $0(x)$ even though $0(c) = 0_{A}$ for each $c \in A$.
			\end{defn}
		% subsection polynomial_functions_and_roots (end)
	% section polynomials_over_a_ring (end)

	\section{Factoring Polynomials}
	\label{sec:factoring_polynomials}
		\subsection{Factors and Irreducible Polynomials}
		\label{sec:factors_and_irreducible_polynomials}
			\setcounter{defn}{0}
			\setcounter{thm}{0}

			\begin{defn}
			Let $A$ be a commutative ring with unity and $a(x), d(x) \in A[x]$. 
			We say that $a(x)$ is a factor of $d(x)$ if there exists a polynomial $b(x) \in A[x]$ with $d(x) = a(x)b(x)$.
			\end{defn}

			\setcounter{defn}{3}
			\begin{defn}
			Let $A$ be an integral domain. 
			Polynomials $a(x),b(x) \in A[x]$ are called associates if there is a nonzero element $c \in A$ so that the constant polynomial $c(x) = c$ has $a(x) = c(x)b(x)$.
			
			We will frequently write $a(x) = cb(x)$ instead of first defining the constant polynomial $c(x) = c$.
			\end{defn}

			\setcounter{thm}{4}
			\begin{thm}
			Let $A$ be an integral domain and suppose $a(x), b(x) \in A[x]$ are associates. 
			Then $c \in A$ is a root of $a(x)$ if and only if $c$ is a root of $b(x)$.
			\end{thm}

			\setcounter{defn}{6}
			\begin{defn}
			Let $A$ be an integral domain with $a(x) \in A[x]$ and $\deg(a(x))>0$.
			We say that $a(x)$ is irreducible over $A$ if every factor of $a(x)$ in $A[x]$ is either a constant polynomial or an associate of $a(x)$.
			If instead a nonconstant factor of $a(x)$ which is not an associate of $a(x)$ exists in $A[x]$, we say that $a(x)$ is reducible over $A$.
			\end{defn}

			\setcounter{thm}{7}
			\begin{thm}
			Let $K$ be a field and suppose $a(x), b(x) \in K[x]$ are associates. 
			The polynomial $a(x)$ is irreducible over $K$ if and only if $b(x)$ is irreducible over $K$.
			\end{thm}

			\begin{thm}
			Let $K$ be a field. 
			Every polynomial in $K[x]$ of degree 1 is irreducible over $K$.
			\end{thm}

			\begin{thm}
			Suppose $K$ is a field, and $p(x) \in K[x]$. 
			If $p(x)$ is irreducible over $K$ then $\pid{p(x)}$ is a maximal ideal of $K[x]$.
			\end{thm}

			\begin{thm}
			Let $K$ be a field, and assume that $p(x) \in K[x]$ is irreducible over $K$. 
			If $a(x), b(x) \in K[x]$ and $p(x)$ is a factor of the product $a(x)b(x)$, then $p(x)$ is a factor of at least one of $a(x)$ or $b(x)$.
			\end{thm}

			\begin{thm}
			Let $K$ and $E$ be fields, and suppose that $\bar{f} : K \to E$ is an \textit{isomorphism}. 
			The polynomial $p(x) \in K[x]$ is irreducible over $K$ if and only if $\pid{p(x)}$ is irreducible over $E$.
			\end{thm}
		% subsection factors_and_irreducible_polynomials (end)

		\subsection{Roots and Factors}
		\label{sec:roots_and_factors}
			\begin{thm}
			Let $K$ be a field and $a(x) \in K[x]$ with $a(x) \neq 0(x)$. 
			The element $c \in K$ is a root of $a(x)$ if and only if $b(x) = -c + x$ is a factor of $a(x)$.
			\end{thm}

			\setcounter{thm}{16}
			\begin{thm}
			Suppose $K$ is a field and $a(x) \in K[x]$ with $a(x) \neq 0(x)$. 
			If the distinct elements $c_{1}, c_{2}, \dots , c_{n} \in K$ are all roots of $a(x)$, then the product $b(x) = (-c_{1} + x)( -c_{2} + x) \cdots (-c_{n} + x)$ is a factor of $a(x)$.
			\end{thm}

			\setcounter{thm}{18}
			\begin{thm}
			Let $K$ be a field. 
			If $c_{1}, c_{2}, \dots , c_{n} \in K$ are distinct roots of the nonzero polynomial $a(x) \in K[x]$, then $\deg(a(x)) \geq n$.
			\end{thm}

			\begin{thm}
			Suppose $K$ is a field and $a(x) \in K[x]$. 
			If $\deg(a(x)) > 0$ then there exist a positive integer $m$ and polynomials $b_{1}(x),b_{2}(x), \dots ,b_{m}(x) \in K[x]$ which are irreducible over $K$ and $a(x) = b_{1}(x)b_{2}(x)\cdots b_{m}(x)$.
			\end{thm}

			\setcounter{thm}{21}
			\begin{thm}
			Let $K$ be a field and $a(x) \in K[x]$ with $\deg(a(x)) = 2$ or $\deg(a(x)) = 3$. 
			The polynomial $a(x)$ is reducible over $K$ if and only if $a(x)$ has a root in $K$.
			\end{thm}

			\setcounter{defn}{23}
			\begin{defn}
			Let $K$ be a field and $a(x) \in K[x]$. 
			Suppose $a(x) \neq 0(x)$, with $\deg(a(x)) = n$. 
			The polynomial $a(x)$ is \textbf{\textit{monic}} if $a_{n} = 1_{K}$.
			\end{defn}

			\setcounter{defn}{25}
			\begin{defn}
			Let $K$ be a field and $a(x) \in K$ with $a(x) \neq 0(x)$. 
			Suppose $c \in K$ is a root of $a(x)$. 
			If there is an integer $m > 0$ for which the polynomial $b(x) = (-c +x)^{m}$ is a factor of $a(x)$ but $d(x) = (-c + x)^{m+1}$ is not a factor of $a(x)$, then we say that $c$ is a root of $a(x)$ with multiplicity $m$.
			\end{defn}

			\setcounter{thm}{26}
			\begin{thm}
			Let $K$ be a field and $a(x) \in K[x]$ with $a(x) \neq 0(x)$. 
			If $\deg(a(x)) = n$ then there can be at most $n$ distinct roots of $a(x)$ in $K$.
			\end{thm}

			\begin{thm}
			Let $K$ be an infinite field. 
			If $a(x),b(x) \in K[x]$, and $a(x) \neq b(x)$, then there must exist some $c \neq K$ for which $a(c) \neq b(c).$
			\end{thm}
		% subsection roots_and_factors (end)

		\subsection{Factorization over $\Q$}
		\label{sec:factorization_over_q}
			\begin{thm}
			If $a(x) \in \Q[x]$ with $a(x) \neq 0(x)$ then there is a polynomial $b(x) \in \Z[x]$ with $\deg(a(x)) = \deg(b(x))$ which has exactly the same rational roots as $a(x)$.
			\end{thm}

			\setcounter{thm}{30}
			\begin{thm}[The Rational Roots Theorem]
			Let $a(x) \in \Z[x]$ with $a(x) \neq 0(x)$ and $\deg(a(x)) = n$. 
			If the rational number $\frac{s}{t} \ (s,t \in \Z$ with no common prime factors and $t \neq 0)$ is a root of $a(x)$ then $s$ must evenly divide $a_{0}$ and $t$ must evenly divide $a_{n}$.
			\end{thm}

			\setcounter{thm}{32}
			\begin{thm}
			If $a(x) \in Z[x]$ and $a(x) = b(x)c(x)$ with $b(x),c(x) \in \Q[x]$, $\deg(b(x)) > 0$, and $\deg(c(x)) > 0$, then there exist polynomials $u(x), w(x) \in \Z[x]$ with $a(x) = u(x)w(x)$, $\deg(u(x)) > 0$, and $\deg(w(x)) > 0$.
			\end{thm}

			\setcounter{thm}{34}
			\begin{thm}[Eisenstein's Criterion]
			Suppose $a(x) \in \Z[x]$ and $\deg(a(x)) = n$ with $n > 0$. 
			If there exists a prime number $p$ which divides coefficients $a_{0}, a_{1}, \dots, a_{n-1}$ but not $a_{n}$, and $p^{2}$ does not divide $a_{0}$, then $a(x)$ is irreducible over $\Q$.
			\end{thm}

			\setcounter{thm}{36}
			\begin{thm}
			Suppose $a(x) \in \Z[x]$ is a monic polynomial and $\deg(a(x)) = k$ with $k > 0$. 
			If there exists $n > 1$ so that $\bar{f}_{n}(a(x))$ is irreducible in $\Z_{n}[x]$ then $a(x)$ is also irreducible in $\Z[x]$.
			\end{thm}
		% subsection factorization_over_q (end)
	% section factoring_polynomials (end)

	\section{Field Extensions}
	\label{sec:field_extensions}
		\setcounter{defn}{0}
		\setcounter{thm}{0}

		\subsection{Extension Field}
		\label{sec:extension_field}
			\begin{defn}
			Suppose that $K$ and $E$ are fields with $K \subseteq E$.
			If for all $a,b \in K$ we have $a +_{K} b = a +_{E} b$ and $a \cdot_{K} b = a \cdot_{E} b$, then $K$ is a subfield of $E$ or $E$ is an extension field of $K$.
			\end{defn}

			\begin{defn}
			Suppose $E$ is an extension field of $K$, and $c \in E$
				\begin{enumerate}[(i)]
				\item If there exists $a(x) \in K[x]$ with $a(x) \neq 0(x)$ and $a(c) = 0_{E}$, then $c$ is \textbf{\textit{algebraic over $K$}}.
				\item If for \underline{every nonzero} $a(x) \in K[x]$ we have $a(c) \neq 0_{E}$, then $c$ is \textbf{\textit{transcendental over $K$}}.
				\end{enumerate}
			\end{defn}

			\setcounter{thm}{3}
			\begin{thm}
			Suppose $E$ is an extension field of $K$, $a(x) \in K[x]$, and there is $c \in E$ with $a(c) = 0_{E}$.
				\begin{enumerate}[(i)]
				\item If $\deg(a(x)) = 1$, then $c \in K$.
				\item If $a(x)$ is irreducible over $K$ and $\deg(a(x)) > 1$, then $c \not\in K$
				\end{enumerate}
			\end{thm}

			\begin{thm}
			Suppose $K$ is a field.
			$E$ is an extension field of $K$ and $c \in E$.
			If $c$ is algebraic over $K$, then there exists a field $K(c)$ (``\textbf{K adjoin c}'') with:
				\begin{enumerate}[(i)]
				\item $K \subseteq K(c) \subseteq E$.
				\item $c \in K(c)$.
				\item For any subfield $S$ of $E$ with $K \subseteq S$ and $c \in S$ we have $K(c) \subseteq S$.
				\end{enumerate}
			\end{thm}

			\setcounter{thm}{6}
			\begin{thm}
			Let $K$ be a field and assume $a(x) \in K[x]$ is irreducible over $K$.
			Then there exists a field $E$ so that $E$ is an extension field of $K$ and $a(x)$ has a root in $E$.
			\end{thm}
		% subsection extension_field (end)

		\subsection{Minimum Polynomial}
		\label{sec:minimum_polynomial}
			\setcounter{thm}{8}
			\begin{thm}
			If $K$ is a field, $E$ is an extension field of $K$, and $c \in E$ is algebraic over $K$, then there is a \textbf{\textit{unique monic}} polynomial $p(x) \in K[x]$ that is irreducible over $K$ and has $c$ as a root.
			\end{thm}

			\setcounter{defn}{9}
			\begin{defn}
			Let $K$ be a field, $E$ and extension field of $K$, and $c \in E$ algebraic over $K$.
			The unique monic polynomial $p(x) \in K[x]$ that is irreducible over $K$ and has $c$ as a root is called \textbf{\textit{the minimum polynomial}} for $c$ over $K$.
			\end{defn}

			\setcounter{thm}{11}
			\begin{thm}
			Suppose $K$ be a field, $E$ and extension field of $K$, and $c \in E$ algebraic over $K$ with minimum polynomial $p(x) \in K[x]$.
				\begin{enumerate}[(i)]
				\item Using the homomorphism $f_{c} : K[x] \to E$ as defined in Theorem 9.5, $\ker{(f_{c})} = \pid{p(x)}$.
				\item If $b(x) \in K[x]$ is a nonzero polynomial with $b(c) = 0_{E}$, then $b(x) = p(x)q(x)$ for some $q(x) \in K[x]$.
				\end{enumerate}
			\end{thm}

			\begin{thm}
			Suppose $K$ be a field, $E$ and extension field of $K$, and $c \in E$ algebraic over $K$.
			If $p(x)$ is the minimum polynomial for $c$ over $K$, and $\deg(p(x)) = n$, then: $$K(c) = \{ a(c) : a(x) \in K[x] \ \text{and either } a(x) = 0(x) \ \text{or } \deg(a(x)) < n \}.$$
			\end{thm}
		% subsection minimum_polynomial (end)

		\subsection{Algebraic Extensions}
		\label{sec:algebraic_extensions}
			\setcounter{defn}{15}
			\begin{defn}
			Let $K$ be a field and $E$ an extension field of $K$.
			If every element of $E$ is algebraic over $K$ we say that $E$ is an \textbf{algebraic extension} of $K$.
			\end{defn}

			\begin{defn}
			Let $K$ be a field and $E$ an extension field of $K$.
			A nonempty subset of $E$, $B = \{ u_{1}, u_{2}, \dots, u_{m} \}$ is called a \textbf{basis for E over K} when the following hold:
				\begin{enumerate}[(i)]
				\item For every element $s \in E$ there exist $a_{1}, a_{2}, \dots, a_{m} \in K$ so that $s = a_{1}u_{1} + a_{2}u_{2} + \dots a_{m}u_{m}$ ($B$ spans $E$ over $K$).
				\item If $a_{1}, a_{2}, \dots, a_{m} \in K$ with $a_{1}u_{1} + a_{2}u_{2} + \dots a_{m}u_{m} = 0_{E}$ then $a_{i} = 0_{K}$ for all $i = 1, \dots, m$ ($B$ is independent over $K$).
				\end{enumerate}
				
			If there exist $m$ elements of $E$ that form a basis for $E$ over $K$ we say $E$ is a \textbf{finite extension} of $K$ of degree $m$, and write $[E:K] = m$.
			\end{defn}

			\setcounter{thm}{18}
			\begin{thm}
			Let $K$ be a field and $E$ an extension field of $K$.
				\begin{enumerate}[(i)]
				\item Every basis for $E$ over $K$ has the same cardinality.
				\item Every subset of $E$ that spans $E$ contains a basis for $E$ over $K$.
				\end{enumerate}
			\end{thm}

			\begin{thm}
			Suppose $K$ is a field, $c$ is algebraic over $K$ with minimum polynomial $p(x)$, and $\deg(p(x)) = n$.
			Then the set $B = \{ 1_{K}, c, c^{2}, \dots, c^{n-1} \}$ is a bsis for $K(c)$ over $K$ and $[K(c):K]=\deg(p(x))$.
			\end{thm}

			\begin{thm}
			Let $K$ be a field and $E$ an extension field of $K$ with $[E:K] = n$ for some $n>0$.
			Then $E$ is an algebraic extension of $K$.
			\end{thm}

			\begin{thm}
			Suppose that $K$ is a field and $L$ is a finite extension of $K$.
			If $E$ is a finite extension of $L$, then $E$ is also a finite extension of $K$ and $[E:K] = [E:L][L:K]$.
			\end{thm}
		% subsection algebraic_extensions (end)

		\subsection{Root Field of a Polynomial}
		\label{sec:root_field_of_a_polynomial}
			\setcounter{defn}{22}
			\begin{defn}
			Let $K$ be a field and $a(x) \in K[x]$ have $\deg(a(x)) > 0$.
			The \textbf{root field for $a(x)$ over $K$} is a field extension $E$ of $K$ with the following properties:
				\begin{enumerate}[(i)]
				\item In $E[x], a(x)$ can be factored into a product of polynomials of degree 1.
				\item For any extension of $K$, $L$, which satisfies (i), we have $K \subseteq E \subseteq L$.
				\end{enumerate}
			\end{defn}

			\setcounter{defn}{24}
			\begin{defn}
			Let $K$ be a field and $c_{1}, c_{2}$ algebraic over $K$.
			Let $L = K(c_{1})$, then the field $K(c_{1},c_{2}) = L(c_{2})$ is called the \textbf{iterated extension} of $K$.
			\end{defn}

			\setcounter{thm}{25}
			\begin{thm}
			Let $K$ be a field and $a(x) \in K[x]$ with $\deg(a(x)) > 0$.
			If $E$ is the root field of $a(x)$ over $K$, and the elements $c_{1}, c_{2}, \dots, c_{n} \in E$ are all of the distinct roots of $a(x)$, then $E = K(c_{1}, c_{2}, \dots, c_{n})$.
			\end{thm}

			\setcounter{defn}{27}
			\begin{defn}
			Suppose that $E$ is an extension field of $K$ with $c \in E$.
			A field extension $K(c)$ is called a \textbf{simple extension} of $K$.
			\end{defn}

			\begin{defn}
			Let $K$ be a field and $p(x) \in K[x]$.
			We say $p(x)$ is \textbf{separable} if no irreducible factor of $p(x)$ has multiple roots in any extension field of $K$.
			Otherwise, we say $p(x)$ is \textbf{inseparable}.
			\end{defn}

			\setcounter{thm}{30}
			\begin{thm}
			Let $K$ be a field with $\char{K} = 0$.
			Then every irreducible polynomial in $K[x]$ is separable.
			\end{thm}

			\begin{thm}
			Let $K$ be a finite field with $\char{K} = q$ for some prime $q$.
				\begin{enumerate}[(i)]
				\item For any polynomial $b(x) = b_{0} + b_{1}x + \cdots + b_{t}x^{t} \in K[x]$ we have $(b(x))^{q} = b_{0}^{q} + b_{1}^{q}x^{q} + \cdots + b_{t}^{q}x^{qt}$.
				\item For any element $s \in K$ there is $r \in K$ with $s = r^{q}$.
				\end{enumerate}
			\end{thm}

			\begin{thm}
			Let $K$ be a finite field.
			Then every irreducible polynomial in $K[x]$ is separable.
			\end{thm}

			\begin{thm}
			Let $K$ be a field of characteristic 0, and $E$ a finite extension of $K$.
			Then $E$ is a simple extension of $K$, meaning there is some $c \in E$ with $E = K(c)$.
			\end{thm}
		% subsection root_field_of_a_polynomial (end)
	% section field_extensions (end)

	\section{Galois Theory}
	\label{sec:galois_theory}
		\setcounter{defn}{0}
		\setcounter{thm}{0}

		\subsection{Isomorphisms and Extension Fields}
		\label{sec:isomorphisms_and_extension_fields}
			\setcounter{defn}{1}
			\begin{defn}
			Let $K$ be a field and $f : K \to K$ be a function.
			If $f$ is an isomorphism we say that $f$ is an \textbf{automorphism of K}.
			\end{defn}

			\begin{defn}
			Let $K$ be a field and $E_{1}, E_{2}$ be exention fields of $K$.
			Suppose $f : E_{1} \to E_{2}$ is an isomorphism.
			If for every $a \in K$ we have $f(a) = a$, then we say that \textbf{f fixes K}.
			\end{defn}

			\setcounter{thm}{3}
			\begin{thm}
			Suppose $K_{1}, K_{2}$ are fields, $f : K_{1} \to K_{2}$ is an isomorphism, and $p(x) \in K_{1}[x]$ is irreducible over $K_{1}$.
			Then there exist extension fields $K_{1}(c_{1})$ and $K_{2}(c_{2})$ with the following properties:
				\begin{enumerate}[(i)]
				\item $c_{1}$ is a root of $p(x)$ and $c_{2}$ is a root of $\bar{f}(p(x))$ (as defined in Theorem 7.28).
				\item There exists an isomorphism $g : K_{1}(c_{1}) \to K_{2}(c_{2})$ with $g(c_{1}) = c_{2}$ for any $a \in K_{1}$, $g(a) = f(a)$.
				\end{enumerate}
			\end{thm}

			\setcounter{thm}{6}
			\begin{thm}
			Let $K$ be a field and $p(x) \in K[x]$ an irreducible polynomial.
			If $c_{1}$ and $c_{2}$ are roots of $p(x)$ in some extension of $K$, then $K(c_{1}) \cong K(c_{2})$ where the isomorphism $g : K(c_{1}) \to K(c_{2})$ maps $g(c_{1}) = c_{2}$ and fixes $K$.
			\end{thm}

			\setcounter{thm}{8}
			\begin{thm}
			Let $K$ be any field and $E_{1}, E_{2}$ extension fields of $K$, with $f : E_{1} \to E_{2}$ an isomorphism fixing $K$.
			If $p(x) \in K[x]$ and $c \in E_{1}$ is a root of $p(x)$, then $f(c) \in E_{2}$ is also a root of $p(x)$.
			\end{thm}
		% subsection isomorphisms_and_extension_fields (end)

		\subsection{Automorphisms of Root Fields}
		\label{sec:automorphisms_of_root_fields}
			\begin{thm}
			Let $K$ be a field and $a(x) \in K[x]$.
			If $\deg(a(x)) = n > 0$, then $a(x)$ is \underline{exactly} $n$ roots in its root field.
			\end{thm}

			\setcounter{thm}{11}
			\begin{thm}
			Suppose that $K$ is a field and $E_{1}, E_{2}$ are both finite exentions of $K$.
			If there exists an isomorphism $f : E_{1} \to E_{2}$ which fixes $K$, and $E_{1}$ is a root field of the polynomial $p(x) \in K[x]$, then $E_{1} = E_{2}$.
			\end{thm}

			\begin{thm}
			Suppose $K$ is a field and $E$ is the root field for some nonconstant $p(x) \in K[x]$.
			Suppose $L_{1}$ and $L_{2}$ are finite extension fields of $K$ with $K \subseteq L_{1} \subseteq E$.
			If there exists $f : L_{1} \to L_{2}$ an isomorphism fixing $K$, then $L_{2} \subseteq E$ and there exist an automorphism $g$ of $E$, with $g(a) = f(a)$ for all $a \in L$.
			\end{thm}

			\begin{thm}
			Suppose $K$ is a field, $E$ is the root field of a polynomial in $K[x]$, and $p(x) \in K[x]$ is irreducible over $K$ with $\deg(p(x)) > 1$.
			For any two distinct roots $c_{1},c_{2} \in E$ of $p(x)$, there exists an automorphism of $E$ fixing $K$, mapping $c_{1}$ to $c_{2}$.
			\end{thm}

			\setcounter{thm}{15}
			\begin{thm}
			Suppose $K$ is a field and $E$ is the root field of a polynomial in $K[x]$.
			If the irreducible polynomial $a(x) \in K[x]$ has one root in $E$ then every root of $a(x)$ is in $E$.
			\end{thm}
		% subsection automorphisms_of_root_fields (end)

		\subsection{The Galois Group of a Polynomial}
		\label{sec:the_galois_group_of_a_polynomial}
			\setcounter{thm}{17}
			\begin{thm}
			Let $K$ be a field and $p(x) \in K[x]$.
			If $E$ is the root field of $p(x)$ over $K$ then the set pf all automorphisms of $E$ fixing $K$ is a group under composition.
			\end{thm}

			\setcounter{defn}{18}
			\begin{defn}
			Let $K$be a field and $p(x) \in K[x]$ with root field $E$.
			The group of automorphisms of $E$ fixing $K$ is called the \textbf{Galois group of E over K}, denoted $Gal(E/K)$.
			It can also be called the \textbf{Galois group of p(x) over K}.
			\end{defn}

			\setcounter{thm}{19}
			\begin{thm}
			Let $K$ be a field and $E$ the root field for some $p(x) \in K[x]$.
			The number of automorphisms of $E$ fixing $K$ is equal to $[E:K]$.
			\end{thm}

			\setcounter{thm}{22}
			\begin{thm}
			Let $K$ be a field and $p(x) \in K[x]$ with root field $E$.
			Let $G = Gal(E/K)$.
			If $H$ is a subgroup of $G$ then the set $E_{H} = \{ y \in E : \alpha(y) = y \ \text{for every } \alpha \in H \}$ is a subfield of $E$ and $K \subseteq E_{H} \subseteq E$.
			The field $E_{H}$ is called \textbf{the fixed field for H}.
			\end{thm}

			\begin{thm}
			Let $K$ be a field, $p(x) \in K[x]$ with root field $E$, and $G = Gal(E/K)$.
			Ifn $L$ is a subfield of $E$ with $K \subseteq L$, then $G_{L} = \{ \alpha \in G : \ \text{for every } y \in L, \alpha(y) = y \}$ is a subgroup of $G$.
			The subgroup $G_{L}$ is called \textbf{the fixer of L}.
			\end{thm}
		% subsection the_galois_group_of_a_polynomial (end)

		\subsection{The Galois Correspondence}
		\label{sec:the_galois_correspondence}
			\begin{thm}
			Let $K$ be a field, $p(x) \in K[x]$ with root field $E$, and $G = Gal(E/K)$.
			If $L$ is a subfield of $E$ with $K \subseteq L \subseteq E$, then $L$ is the fixed field of $G_{L}$, i.e., $E_{G_{L}} = L$.
			\end{thm}

			\setcounter{thm}{26}
			\begin{thm}
			Let $K$ be a field, $p(x) \in K[x]$ with root field $E$, and $G = Gal(E/K)$.
			If $H$ is a subgroup of $G$ then the fixer of the field $E_{H}$ is $H$, i.e., $G_{E_{H}} = H$.
			\end{thm}

			\setcounter{thm}{28}
			\begin{thm}
			Let $K$ be a field and $E$ the root field for a polynomial over $K$.
			If $L$ is a subfield of $E$ with $K \subseteq L$ and $L$ is a root field over $K$, then $Gal(E/L) \triangleleft Gal(E/K)$ and $Gal(L/K) \cong Gal(E/K) / Gal(E/L)$.
			\end{thm}

			\begin{thm}
			Let $K$ be a field, $E$ the root field for a polynomial over $K$, and $L$ an intermediate field $K \subseteq L \subseteq E$.
			If $Gal(E/L) \triangleleft Gal(E/K)$ then $L$ is a root field over $K$.
			\end{thm}
		% subsection the_galois_correspondence (end)
	% section galois_theory (end)

	\section{Solvability}
	\label{sec:solvability}
		\setcounter{defn}{0}
		\setcounter{thm}{0}

		\subsection{Three Construction Problems}
		\label{sec:three_construction_problems}
			There are no definitions or theorems in this subsection.
		% subsection three_construction_problems (end)

		\subsection{Solvable Groups}
		\label{sec:solvable_groups}
			\begin{defn}
			A group $G$ is called a \textbf{solvable group} if there are subgroups $\{ e_{G} \} = H_{0}, H_{1}, \dots, H_{n} = G$, so that for each $0 \leq i \leq n - 1$, $H_{i} \triangleleft H_{i+1}$ and $H_{i+1} / H_{i}$ is an abelian group. 
			\end{defn}

			\setcounter{thm}{2}
			\begin{thm}
			The permutation group $S_{5}$ is not solvable.
			\end{thm}

			\begin{thm}
			The permutation group $S_{4}$ is not solvable.
			\end{thm}

			\begin{thm}
			Let $G$ be a group and $J$ a subgroup of $G$.
				\begin{enumerate}[(i)]
				\item If $G$ is a solvable group then $J$ is a solvable group.
				\item If $J \triangleleft G$ and both $J$ and $G/J$ are solvable groups, then $G$ is a solvable group.
				\end{enumerate}
			\end{thm}

			\begin{thm}
			Suppose $G$ and $B$ are groups.
			If there is an \underline{onto} homomorphism $f : G \to B$ and $G$ is a solvable group, then $B$ is a solvable group.
			\end{thm}

			\begin{thm}
			For each $n \geq 5$, $S_{n}$ is not a solvable group.
			\end{thm}
		% subsection solvable_groups (end)

		\subsection{Solvable by Radicals}
		\label{sec:solvable_by_radicals}
			\setcounter{defn}{8}
			\begin{defn}
			Let $K$ be a field.
			A \textbf{radical extension} of $K$ is a finite extension of the form $K(c_{1},\dots,c_{n})$ where for each $1 \leq i \leq n$, there is a positive integer $m_{i} \geq 2$ so that $(c_{1})^{m_{1}} \in K$ and for $1 < i \leq n$, $(c_{i})^{m_{i}} \in K(c_{1},\dots,c_{i-1})$
			\end{defn}

			\setcounter{defn}{10}
			\begin{defn}
			Let $K$ be a field and $p(x) \in K[x]$.
			We say that $p(x)$ is \textbf{solvable by radicals} if the root field of $p(x)$ is contained in a radical extension of $K$.
			\end{defn}

			\setcounter{thm}{12}
			\begin{thm}
			Let $L$ be a radical extension of $\Q$.
			Then there exists a radical extension $E$ of $\Q$, with $\Q \subseteq L \subseteq E$, where $E$ is also a root field over $\Q$.
			\end{thm}

			\setcounter{defn}{13}
			\begin{defn}
			For $n > 1$, the root $\omega_{n} = \cos{\left( \frac{2\pi}{n} \right)} + i \sin{\left( \frac{2\pi}{n} \right)}$ for the polynomial $-1 + x^{n} \in \Q[x]$ is called a \textbf{primitive n$^{th}$ root of unity}.
			\end{defn}

			\setcounter{thm}{14}
			\begin{thm}
			For each positive integer $n$, the polynomial $p(x) = -1 + x^{n}$ in $\Q[x]$ is solvable by radicals.
			\end{thm}

			\begin{thm}
			If $n$ is a positive integer with $n \geq 2$ then $Gal(\Q(\omega_{n})/\Q)$ is abelian.
			\end{thm}

			\begin{thm}
			Let $m_{1}, m_{2}, \dots,m_{r}$ be distinct positive integers, and $m_{j} \geq 2$ for each $j$.
			Then the field $L = \Q(\omega_{m_{1}}, \omega_{m_{2}}, \dots, \omega_{m_{r}})$ is a root field over $\Q$, and $Gal(L/\Q)$ is a solvable group.
			\end{thm}

			\begin{thm}
			If the polynomial $p(x) \in \Q[x]$ is solvable by radicals, then the Galois group $G$ for $p(x)$ is a solvable group.
			\end{thm}

			\setcounter{thm}{19}
			\begin{thm}
			If $G$ is a \underline{finite} solvable group, then there is a sequence of subgroups $\{ e_{G} \} = H_{0},H_{1},\dots,H_{n} = G$ where for each $0 \leq j < n$, $H_{j} \triangleleft H_{j+1}$, and $H_{j+1}/H_{j}$ is cyclic of prime order.
			\end{thm}

			\begin{thm}
			If $p(x) \in \Q[x]$ has root field $E$ and $Gal(E/\Q)$ is solvable, then $p(x)$ is solvable by radicals.
			\end{thm}
		% subsection solvable_by_radicals (end)
	% section solvability (end)

	\section{Constructible Numbers}
	\label{sec:constructible_numbers}
		\setcounter{defn}{21}
		\begin{defn}
			We say that a real number $\alpha$ is constructible if -- using an unmarked straight-edge and compass -- we can build a line segment of length $|\alpha|$ using the following geometric operations:
			\begin{enumerate}[\hspace*{2.5mm}(i)]
				\item Given a constructed point $P$ and constructed line $\ell$, we can construct a unique line $\ell'$ through $P$ that is perpendicular to $\ell$.
				\item Given a constructed point $P$ and constructed line $\ell$, we can construct a unique line $\ell''$ through $P$ that is parallel to $\ell$.
				\item Give a constructed point $P$ and constructed length $|\alpha|$, we can construct a point $Q$ on $\ell$ such that $PQ = |\alpha|$.
			\end{enumerate}
		\end{defn}

		\setcounter{thm}{22}
		\begin{thm}
			The set of constructable numbers is a field extension of $\mathbb{Q}$ and is closed under taking square roots.
		\end{thm}

		\begin{thm}
			Let $\alpha$ be a constructible number.
			Then there exists a sequence (`tower') of finite field extensions $\Q = F_{0} \subset F_{1} \subset \cdots \subset F_{n}$ such that:
			\begin{enumerate}[\hspace*{2.5mm}(i)]
				\item $F_{n} \subset \R$.
				\item $\alpha \in F_{n}$.
				\item For each $i$, we have $F_{i} = F_{i}(\sqrt{r_{i}})$ where $r_{i} \in F_{i}$.
			\end{enumerate}
			Conversely, given a sequence of field extensions satisfying conditions $(i)$ and $(iii)$, then all $x \in F_{n}$ are constructible.
		\end{thm}

		\setcounter{cor}{24}
		\begin{cor}
			If $\alpha$ is a constructible number, then $[\Q(\alpha):\Q] = 2^{n}$ for some $n \in \N$.
		\end{cor}

		\setcounter{thm}{25}
		\begin{thm}
			Using an unmarked straight-edge and compass, it is impossible to construct a cube of volume two units.
		\end{thm}

		\begin{thm}
			It is impossible to trisect an angle of 60 degrees using an unmarked straight-edge and compass.
		\end{thm}

		\setcounter{cor}{27}
		\begin{cor}
			Using an unmarked straight-edge and compass, it is impossible to construct an angle of 20 degrees.
		\end{cor}
	% section constructible_numbers (end)
\end{document}