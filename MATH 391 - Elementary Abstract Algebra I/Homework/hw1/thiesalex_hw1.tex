%!TEX output_directory = temp
\documentclass{amsart}
	%% Basic Info
		\author{Alex Thies}
		\title{Homework 1 \\ Math 391 - Fall 2017}
		%\email{athies@uoregon.edu}
	%% boilerplate packages
		\usepackage{amsmath}
		\usepackage{amssymb}
		\usepackage{amsthm}
		\usepackage{mathrsfs}
		\usepackage{enumerate}
	%% pretty letters
		\DeclareMathOperator{\ep}{\varepsilon}
		\DeclareMathOperator{\ph}{\varphi}
		\DeclareMathOperator{\Ell}{\mathscr{L}}
	%% congruences
		\renewcommand{\mod}[1]{\ (\mathrm{mod}\ #1)}
	%% sets
		\DeclareMathOperator{\N}{\mathbb{N}}
		\DeclareMathOperator{\Z}{\mathbb{Z}}
		\DeclareMathOperator{\Zp}{\mathbb{Z}^{+}}
		\DeclareMathOperator{\Zm}{\mathbb{Z}^{-}}
		\DeclareMathOperator{\Q}{\mathbb{Q}}
		\DeclareMathOperator{\Qc}{\mathbb{Q}^{c}}
		\DeclareMathOperator{\R}{\mathbb{R}}
		\DeclareMathOperator{\C}{\mathbb{C}}
		\DeclareMathOperator{\pset}{\mathcal{P}}
\begin{document}
	\maketitle

	\begin{enumerate}
		\item[\textbf{Problem 0.9.}] For the sets $A$, $B$, and $C$ in Exercises 0.1, 0.2, 0.3, find the sets $A\cup(C \cap B)$, $\pset(A)$.
		\begin{proof}[Solution] The sets from 0.1, 0.2, and 0.3 are 
			\begin{align*}
				A &= \{x : \text{$x$ is an integer and $-2 < x < 3$}\} \\
				B &= \{x : \text{$x$ is a real number and $x^{2}-5x = 0$}\} \\
				C &= \{x : \text{$x$ is an integer and $2x - 6 = 2$}\}
			\end{align*}
			We will determine $A\cup(C \cap B)$.
			Note that $A = (-1, 2)\cap\Z$, $B = (0,5)\cap\R$, and that $C = \{ 4 \}$, thus,
			\begin{align*}
				A \cup (B \cap C) &= \left[ (-1, 2)\cap\Z \right] \cup \left[ \left( (0,5)\cap\R \right) \cap \{ 4 \} \right], \\
				&= \left[ (-1, 2)\cap\Z \right] \cup \{4\}, \\
				&= \{x : \text{$x$ is an integer and $-2 < x < 3$, or $x=4$}\}, \\
				&= \{-1, 0, 1, 2, 4\}.
			\end{align*}
			We will now find $\pset(A)$.
			Recall that $A = \{ -1, 0, 1, 2 \}$.
			Notice that $|A| = 4$, so $|\pset(A)|$ should be $2^{|A|} = 2^{4} = 16.$
			Thus $\pset(A) =$ $\{ \{\},$ $\{-1\},$ $\{0\},$ $\{1\},$ $\{2\},$ $\{-1,0\},$ $\{-1,1\},$ $\{-1,2\},$ $\{0,1\},$ $\{0,2\},$ $\{1,2\},$	$\{-1,0,1\},$ $\{-1,0,2\},$ $\{-1,1,2\},$ $\{0,1,2\},$ $\{-1,0,1,2\} \}$.
			Note that we have 16 elements in $\pset(A)$, so (barring errors of order or repitition) this should be all subsets of $A$.
		\end{proof}

		\item[\textbf{Problem 0.18.}] Prove that the statement ``if $A \cap B \subset C$ then $A \subset C$ or $B \subset C$'' is true or find a counterexample with nonempty sets that makes it fail.
		\begin{proof} We provide the following counterexample. 
		Suppose $A = \{a : a = 2^{n}, n=0,1,2,\dots \}$, $B = \{b : b = 3^{n}, n=0,1,2,\dots \}$, and $C = \{ 1 \}$.
		Then $A\cap B = \{ 1 \}$ as well, but observe that while $A\cap B \subseteq C$, $A \not\subset C$ and $B \not\subset C$.
		\end{proof}

		\item[\textbf{Problem 0.21.}] Prove that the statement ``$A\cap (B \cup C) = (A\cap B)\cup(A \cap C)$'' is true or find a counterexample with nonempty sets that makes it fail.
		\begin{proof} We proceed directly,
			\begin{align*}
				A\cap(B \cup C) &= A \cap \{x : x \in B \text{ or }x \in A \}, \\
				&= \{ x : x \in B \text{ or }x \in A, \text{ and } x \in A \}, \\
				&= \{ x : x \in B \text{ and } x\in A \text{ or, }x \in C \text{ and } x \in A \}, \\
				&= (A \cap B) \cup (A\cap C).
			\end{align*}
		\end{proof}
		\newpage

		\item[\textbf{Problem 0.26.}] Prove: For all $n \in \N, 1^{3} + 2^{3} + \cdots + n^{3} = \frac{n^{2}(n + 1)^{2}}{4}$.
		\begin{proof} We proceed by mathematical induction on the natural numbers.
			\begin{enumerate}
				\item[\textbf{Base case:}] Note that for $n=1$, $1^{3}= 1^{2}(2)^{2}/4$.
				\item[\textbf{Induction Hypothesis:}] Assume that $\sum_{i=1}^{n} i^{3} = n^{2}(n+1)^{2}/4$ for some $n \in \N$.
				\item[\textbf{Induction Step:}] We compute the following,
					\begin{align*}
						1^{3} + 2^{3} + \cdots + (n+1)^{3} &= 1^{3} + 2^{3} + \cdots + n^{3} + (n+1)^{3} \\
						&= \sum_{i=1}^{n} i^{3} + (n+1)^{3}, \\
						&= \frac{n^{2}(n+1)^{2}}{4} + (n+1)^{3}, \\
						&= (n+1)^{2} \left( \frac{n^{2}}{4} + (n+1) \right), \\
						&= \frac{(n+1)^{2}}{4} \left( n^{2} + 4(n+1) \right), \\
						&= \frac{(n+1)^{2}}{4} \left( n^{2} + 4n + 4) \right), \\
						&= \frac{(n+1)^{2}}{4} (n+2)^{2}, \\
						&= \frac{(n+1)^{2}(n+2)^{2}}{4}.
					\end{align*}
					Hence, we have shown for the given proposition $P$, that $P_{n} \Rightarrow P_{n+1}$.
			\end{enumerate}
		\end{proof}

		\item[\textbf{Problem 0.27.}] Prove: For all $n \in \N$, 4 evenly divides $3^{2n-1} + 1$.
		\begin{proof} We proceed by mathematical induction on the natural numbers.
			\begin{enumerate}
				\item[\textbf{Base case:}] Note that for $n=1$, $3^{2-1}+1 = 4 \equiv 0 \mod{4}$.
				\item[\textbf{Induction Hypothesis:}] Assume that $3^{2n-1}+1 \equiv 0 \mod{4}$ for some $n \in \N$.
				\item[\textbf{Induction Step:}] We compute the following,
					\begin{align*}
						3^{2n-1} + 1 &\equiv 0 \mod{4}, \\
						3^{2n-1} &\equiv -1 \mod{4}, \\
						3^{2n-1} &\equiv -9 \mod{4}, \\
						9\cdot 3^{2n-1} &\equiv -9 \cdot 9 \mod{4}, \\
						3^{2n+1} &\equiv -1 \mod{4}, \\
						3^{2n+1} + 1 &\equiv 0 \mod{4}.
					\end{align*}
					Hence, we have shown for the given proposition $P$, that $P_{n} \Rightarrow P_{n+1}$.
			\end{enumerate}
		\end{proof}

		\item[\textbf{Problem 0.30.}] Define $\sim$ on $\mathbb{Z}$ by $a \sim b$ if and only if $a = 2b$. 
		Either prove that $\sim$ is an equivalence relation or find a counterexample showing it fails.
		\begin{proof} Observe that $\sim$ is not reflexive upon inspection and thus not an equivalence relation. 
		Counterexample, $(1,1) \Rightarrow 1 = 2$ which is absurd.
		\end{proof}

		\item[\textbf{Problem 0.37.}] Define $\sim$ on $\R$ by $x \sim y$ if and only if $x - y \in \Z$. 
		Either prove that $\sim$ is an equivalence relation or find a counterexample showing it fails.
		\begin{proof} Let $(a,b)$ denote $a \sim b$.
		We will show that $\sim$ is reflexive, symmetric, and transitive:
			\begin{enumerate}[(i)]
				\item Note that $x-x = 0$, and that $0 \in \Z$, thus $x-x\in\Z$, hence $(x,x)$, which we aimed to show.
				\item Suppose $x - y = j$ where $j \in \Z$, thus $y - x = -j$, hence $(x,y) = (y,x)$, which we aimed to show.
				\item Assume that $(x,y)$ and $(y,z)$. Then $x - y = k$ and $y - z = \ell$, it follows that $x - z = k + \ell$; because $\Z$ is closed under addition $k + \ell \in \Z$, therefore $(x,y),(y,z) \Rightarrow (x,z)$, which we aimed to show.
			\end{enumerate}
		Hence, because $\sim$ is reflexive, symmetric, and transitive, it is an equivalence relation on $\R$.
		\end{proof}

		\item[\textbf{Problem 0.40.}] Determine whether the function $r: \Z \to \Z$, where $r(x) = 7x$ is injective, surjective, or bijective.
		\begin{proof} We will show that $r$ is a bijection on $\Z$ by proving that it is injective and surjective. 
		For $r$ to be injective, it must be true that if $x_{1} \neq x_{2}$, then $r(x_{1}) \neq r(x_{2})$, or the more useful contrapositive of the previous statement: if $r(x_{1}) = r(x_{2})$, then $x_{1} = x_{2}$. 
		Suppose that $r(x_{1}) = r(x_{2})$ for arbitrary $x_{1}, x_{2} \in \Z$. 
		Then $7x_{1} = 7x_{2}$, therefore $x_{1} = x_{2}$, which shows that $r$ is injective. 
		For $r$ to be surjective it must be true that for all $x_{3} \in \Z$, there exist $x_{4} \in \Z$ such that $r(x_{4}) = x_{3}$. 
		Suppose by way of contradiction that there exists $x_{4} \in \Z$ such that $r(x_{4}) \neq x_{3}$, but given the mapping of $r$, as well as its domain, codomain, and image, no such $x_{4}$ can exist. 
		Thus $r$ maps to every element of the codomain an element of the domain, hence it is surjective. 
		Given that we have also show $r$ to be injective, way may now claim that it is bijective.
		\end{proof}

		\item[\textbf{Problem 0.41.}] Determine whether the function $t: \Q \to \Q$, where $t(x) = 5x - 3$ is injective, surjective, or bijective.
		\begin{proof} We will show that $t$ is a bijection on $\Q$ by proving that it is injective and surjective. 
		For $t$ to be injective, it must be true that if $x_{1} \neq x_{2}$, then $t(x_{1}) \neq t(x_{2})$, or the more useful contrapositive of the previous statement: if $t(x_{1}) = t(x_{2})$, then $x_{1} = x_{2}$. 
		Suppose that $t(x_{1}) = t(x_{2})$ for arbitrary $x_{1}, x_{2} \in \Q$. 
		Then $5x_{1} - 3 = 5x_{2} - 3$, therefore $5x_{1} = 5x_{2}$ and thus $x_{1} = x_{2}$, which shows that $t$ is injective. 
		For $t$ to be surjective it must be true that for all $x_{3} \in \Q$, there exist $x_{4} \in \Q$ such that $t(x_{4}) = x_{3}$. 
		Suppose by way of contradiction that there exists $x_{4} \in \Q$ such that $t(x_{4}) \neq x_{3}$, but given the mapping of $t$, as well as its domain, codomain, and image, no such $x_{4}$ can exist. 
		Thus $t$ maps to every element of the codomain an element of the domain, hence it is surjective. 
		Given that we have also show $t$ to be injective, way may now claim that it is bijective.
		\end{proof}

		\item[\textbf{Problem 0.49.}] Prove: If $f$ and $g$ are functions with $f: A \to B$, $g: B \to C$ and $g$, $f$ are both bijections then $g \circ f$ is a bijection.
		\begin{proof} Given that $f$ and $g$ are bijections, we will show directly that $g \circ f$ is also a bijection; that is, it is injective and surjective. 
		Again utilizing the contrapositive of the definition of an injective function, we will seek to show that $(g \circ f)(x_{1}) = (g \circ f)(x_{2})$ implies that $x_{1} = x_{2}$. 
		Suppose that $(g \circ f)(x_{1}) = (g \circ f)(x_{2})$, recall that $f$ and $g$ are both injective, then
			\begin{align*}
				(g \circ f)(x_{1}) &= (g \circ f)(x_{2}), \\
				g\left( f(x_{1}) \right) &= g\left( f(x_{2}) \right), \\
				g(x_{1}) &= g(x_{2}), \\
				x_{1} &= x_{2}.
			\end{align*}
		Hence, $g \circ f$ is injective; we will now show that it is also surjective. 
		Again, we will proceed directly from the definition. 
		Let $z \in C$, since $g$ is surjective there exists $y \in B$ such that $z = g(y)$. 
		Then, since $f$ is surjective there exists $x \in A$ such that $y = f(x)$. 
		Notice that we have shown that for $z \in C$, there exists $x \in A$ such that $z = (g \circ f)(x)$, hence $g \circ f$ is surjective. 
		We have shown that $g \circ f$ is both injective and surjective, thus it is bijective.
		\end{proof}

		\item[\textbf{Problem 0.58.}] Given the following matrices, compute $CD$ and $DC$.
		\begin{align*}
			C &= \begin{pmatrix}
				1 & 1/3 & -1 \\
				-1 & 0 & 4 \\
				3/4 & 0 & 1
			\end{pmatrix}, & 
			D &= \begin{pmatrix}
				0 & 2/5 & 2 \\
				1 & 1 & 1/2 \\
				3/4 & -1 & 0
			\end{pmatrix}.
		\end{align*}
		\begin{proof}[Solution]
			We compute the following,
				\begin{align*}
					CD &= \begin{pmatrix}
							1 & 1/3 & -1 \\
							-1 & 0 & 4 \\
							3/4 & 0 & 1
						\end{pmatrix}
						\begin{pmatrix}
							0 & 2/5 & 2 \\
							1 & 1 & 1/2 \\
							3/4 & -1 & 0
						\end{pmatrix}, \\
						&= \begin{pmatrix}
						(1 \cdot 0 + 1/3 \cdot 1 - 1 \cdot 3/4) & (1 \cdot 2/5 + 1/3 \cdot 1 + 1 \cdot 1) & (1 \cdot 2 + 1/3 \cdot 1/2 - 1 \cdot 0) \\
						(-1 \cdot 0 + 0 \cdot 1 + 4 \cdot 3/4) & (-1 \cdot 2/5 + 0 \cdot 1 - 4 \cdot 1) & (-1 \cdot 2 + 0 \cdot 1/2 + 4 \cdot 0) \\
						(3/4 \cdot 0 + 0 \cdot 1/2 + 1 \cdot 0) & (3/4 \cdot 2/5 + 1 \cdot 1 - 1 \cdot 1) & (3/4 \cdot 2 + 0 \cdot 1/2 + 1 \cdot 0)
						\end{pmatrix}, \\
						&= \begin{pmatrix}
						-5/12 & 26/15 & 13/6 \\
						3 & -22/5 & -2 \\
						3/4 & -7/10 & 3/2
						\end{pmatrix}. \\ \\
					DC &= \begin{pmatrix}
							0 & 2/5 & 2 \\
							1 & 1 & 1/2 \\
							3/4 & -1 & 0
						\end{pmatrix}
						\begin{pmatrix}
							1 & 1/3 & -1 \\
							-1 & 0 & 4 \\
							3/4 & 0 & 1
						\end{pmatrix}, \\
						&= \begin{pmatrix}
						(0 \cdot 1 - 2/5 \cdot 1 + 2 \cdot 3/4) & (0 \cdot 1/3 + 2/5 \cdot 0 + 2 \cdot 0) & (0 \cdot -1 + 2/5 \cdot 4 + 2 \cdot 1) \\
						(1 \cdot 1 - 1 \cdot 1 + 1/2 \cdot 3/4) & (1 \cdot 1/3 + 1 \cdot 0 + 1/2 \cdot 0) & (1 \cdot -1 + 1 \cdot 4 + 1/2 \cdot 1) \\
						(3/4 \cdot 1 + 1 \cdot 1 + 0 \cdot 3/4) & (3/4 \cdot 1/3 + 1 \cdot 0 + 0 \cdot 0) & (3/4 \cdot 1 - 1 \cdot 4 + 0 \cdot 1)
						\end{pmatrix}, \\
						&= \begin{pmatrix}
						11/10 & 0 & 18/5 \\
						3/8 & 1/3 & 7/2 \\
						7/4 & 1/4 & -19/4
						\end{pmatrix}.
				\end{align*}
		\end{proof}
	\end{enumerate}
\end{document}