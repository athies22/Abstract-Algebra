%!TEX output_directory = temp
\documentclass[letterpaper, 12pt]{amsart}
	%%%%%%%%%%%%%%%%%%%%%%%%%%%%%%%%%%%%%%%%%%%%%%%%%%%%%%%%%%%%%%%%%%%%%%%%%%%%%%
	%%%%%%%%%%%%%%%%%%%%%%%%%%%% boilerplate packages %%%%%%%%%%%%%%%%%%%%%%%%%%%%
	\usepackage{amsmath,amssymb,amsthm}
	\usepackage[mathscr]{euscript}
	\usepackage{enumerate}
	\usepackage{graphicx}
	\usepackage{mathrsfs}
	\usepackage{color}
	\usepackage{hyperref}
	\usepackage{verbatim}
	\usepackage{stmaryrd}
	\usepackage{polynom}
	% \usepackage[margin=1.5in]{geometry}

	%%%%%%%%%%%%%%%%%%%%%%%%%%%%%%%%%%%%%%%%%%%%%%%%%%%%%%%%%%%%%%%%%%%%%%%%%%%%%%
	%%%%%%%%%%%%%%%%%%%%%%%%%%%%% rename the abstract %%%%%%%%%%%%%%%%%%%%%%%%%%%%
	% \renewcommand{\abstractname}{Introduction}

	%%%%%%%%%%%%%%%%%%%%%%%%%%%%%%%%%%%%%%%%%%%%%%%%%%%%%%%%%%%%%%%%%%%%%%%%%%%%%%
	%%%%%%%%%%%%%%%%%%%%%%%%%%%%%%%%%%%%% sets %%%%%%%%%%%%%%%%%%%%%%%%%%%%%%%%%%%
		%% sets 
		\DeclareMathOperator{\N}{\mathbb{N}}
		\DeclareMathOperator{\Z}{\mathbb{Z}}
		\DeclareMathOperator{\Zp}{\mathbb{Z}^{+}}
		\DeclareMathOperator{\Q}{\mathbb{Q}}
		\DeclareMathOperator{\Qp}{\mathbb{Q}^{+}}
		\DeclareMathOperator{\Qc}{\mathbb{Q}^{c}}
		\DeclareMathOperator{\R}{\mathbb{R}}
		\DeclareMathOperator{\Rp}{\mathbb{R}^{+}}
		\DeclareMathOperator{\C}{\mathbb{C}}
		\DeclareMathOperator{\Cnon}{\mathbb{C}^{\times}}
		%% powerset of a set
		\DeclareMathOperator{\pset}{\mathcal{P}}
		%% set of continuous functions in a certain variable
		\DeclareMathOperator{\cont}{\mathscr{C}}
		%% set of functions in a certain variable
		\DeclareMathOperator{\func}{\mathscr{F}}
		
	%%%%%%%%%%%%%%%%%%%%%%%%%%%%%%%%%%%%%%%%%%%%%%%%%%%%%%%%%%%%%%%%%%%%%%%%%%%%%%
	%%%%%%%%%%%%%%%%%%%%%%%%%%%%%%%% linear algebra %%%%%%%%%%%%%%%%%%%%%%%%%%%%%%
		%% linear span
		\DeclareMathOperator{\Ell}{\mathscr{L}}
		%% bold vectors with arrows, and bold matrices
		\newcommand{\bmat}[1]{{\mathbf{#1}}}
		\newcommand{\bvec}[1]{{\vec{\mathbf{#1}}}}
		%% independent vectors/matrices
		\DeclareMathOperator{\ind}{\perp\!\!\!\perp}
		%% order
		\DeclareMathOperator{\ord}{\text{ord}}

	%%%%%%%%%%%%%%%%%%%%%%%%%%%%%%%%%%%%%%%%%%%%%%%%%%%%%%%%%%%%%%%%%%%%%%%%%%%%%%
	%%%%%%%%%%%%%%%%%%%%%%%%%%% probability & statistics %%%%%%%%%%%%%%%%%%%%%%%%%
		%% probability, expectation, variance, etc.
		\renewcommand{\Pr}{\mathbb{P}}
		\DeclareMathOperator{\E}{\mathbb{E}}
		\DeclareMathOperator{\var}{\rm Var}
		\DeclareMathOperator{\sd}{\rm SD}
		\DeclareMathOperator{\cov}{\rm Cov}
		\DeclareMathOperator{\SE}{\rm SE}
		\DeclareMathOperator{\ssreg}{{\rm SS}_{{\rm Reg}}}
		\DeclareMathOperator{\ssr}{{\rm SS}_{{\rm Res}}}
		\DeclareMathOperator{\sst}{{\rm SS}_{{\rm Tot}}}

	%%%%%%%%%%%%%%%%%%%%%%%%%%%%%%%%%%%%%%%%%%%%%%%%%%%%%%%%%%%%%%%%%%%%%%%%%%%%%%
	%%%%%%%%%%%%%%%%%%%%%%%%%%%%%%%% congruences %%%%%%%%%%%%%%%%%%%%%%%%%%%%%%%%%
		\renewcommand{\mod}[1]{\ (\mathrm{mod}\ #1)}

	%%%%%%%%%%%%%%%%%%%%%%%%%%%%%%%%%%%%%%%%%%%%%%%%%%%%%%%%%%%%%%%%%%%%%%%%%%%%%%
	%%%%%%%%%%%%%%%%%%%%%%%%%%%%%% bracket notation %%%%%%%%%%%%%%%%%%%%%%%%%%%%%%
		% I first used this for principal ideals, that is why the abbreviation is pid
		\newcommand{\pid}[1]{\langle #1 \rangle}

	%%%%%%%%%%%%%%%%%%%%%%%%%%%%%%%%%%%%%%%%%%%%%%%%%%%%%%%%%%%%%%%%%%%%%%%%%%%%%%
	%%%%%%%%%%%%%%%%%%%%%%%%%%%%%%% fatdot notation %%%%%%%%%%%%%%%%%%%%%%%%%%%%%%
		\makeatletter
			\newcommand*\fatdot{\mathpalette\fatdot@{.5}}
			\newcommand*\fatdot@[2]{\mathbin{\vcenter{\hbox{\scalebox{#2}{$\m@th#1\bullet$}}}}}
		\makeatother

	%%%%%%%%%%%%%%%%%%%%%%%%%%%%%%%%%%%%%%%%%%%%%%%%%%%%%%%%%%%%%%%%%%%%%%%%%%%%%%
	%%%%%%%%%%%%%%%%%%%%%%%%%%%%%% use pretty letters %%%%%%%%%%%%%%%%%%%%%%%%%%%%
		\DeclareMathOperator{\ep}{\varepsilon}
		\DeclareMathOperator{\ph}{\varphi}

	%%%%%%%%%%%%%%%%%%%%%%%%%%%%%%%%%%%%%%%%%%%%%%%%%%%%%%%%%%%%%%%%%%%%%%%%%%%%%%
	%%%%%%%%%%%%%%%%%%%%%%%%%%%%%%%%%%% theorems %%%%%%%%%%%%%%%%%%%%%%%%%%%%%%%%%
		\newtheorem{lem}{Lemma}
		\DeclareMathOperator{\ra}{\Rightarrow)}
		\DeclareMathOperator{\la}{\Leftarrow)}

\begin{document}
	\title{Quiz 1 Corrections  -- Math 392 \\ \today}
	\author{Alex Thies \\ \href{mailto:athies@uoregon.edu}{\lowercase{athies$@$uoregon.edu}}}

	\maketitle

	\section*{Problem 1}
	\label{sec:problem_1}
	Let $T : \Q[x] \to \R$ be the ring homomorphism given by $$T(p(x)) = p(\sqrt{2}).$$
	We proved last term that every kernel is an ideal, and moreover, we know that any ideal of $\Q[x]$ is a principal ideal, so we know $\ker{T} = \pid{a(x)}$ for some $a(x) \in \Q[x]$.
	Determine this polynomial $a(x)$ (and of course, prove your claim is true).

	\begin{proof}
	We claim that $p(x) = x^{2} - 2$, we will show that $\ker{T} = \pid{x^{2} - 2}$ by double-inclusion.

	Notice that $T(x^{2} - 2) = 0$, thus $x^{2} - 2 \in \ker{T}$, it follows by the zero product property that any linear combination of $x^{2} - 2$, i.e. any element of $\pid{x^{2} - 2}$ will also equal zero when evaluated at $x = \sqrt{2}$.
	Thus, $\pid{x^{2} - 2} \subseteq \ker{T}$.
	It remains to show that $\pid{x^{2} - 2} \supseteq \ker{T}$.

	Let $b(x) \in \ker{T}$ be arbitrary, then $T(b(x)) = b(\sqrt{2}) = 0$.
	By the division algorithm we can write $b(x) = q(x)(x^{2} - 2) + r(x)$ for unique polynomials $q(x), r(x) \in \Q[x]$ such that $\deg{r(x)} < \deg{x^{2} - 2}$, or $r(x) = 0$.
	The remainder of the proof pertains to the nature of $(x)$.
	For $\ker{T}$ to be a subset of $\pid{x^{2} - 2}$ we must have $r(x) = 0(x)$, which we will prove by cases.
	Since $\deg{x^{2} - 2} = 2$, we have either $\deg{r(x)} = 1$, $\deg{r(x)} = 0$, or $r(x) = 0(x)$.

	Case 1, suppose $\deg{r(x)} = 1$, then $r(x) = r_{0} + r_{1}x$ and $r_{1} \neq 0$.
	We can see that $r(\sqrt{2}) \neq 0$, which would contradict our hypothesis that $b(x) \notin \ker{T}$.
	Hence, $\deg{r(x)} \neq 1$.

	Case 2, suppose $\deg{r(x)} = 0$, then $r(x) = r_{0}$ and $r_{0} \neq 0$.
	We can see that $r(\sqrt{2}) \neq 0$, which would contradict our hypothesis that $b(x) \notin \ker{T}$.
	Hence, $\deg{r(x)} \neq 0$.

	This leaves us with the case that $r(x) = 0(x)$, which allows us to write $b(x) = q(x)(x^{2} - 2)$, and then conclude that $b(x) \in \pid{x^{2} - 2}$.
	Thus, we have shown that an arbitrary element of the kernel of $T$ is inherently also an element of $\pid{x^{2} - 2}$, therefore $\ker{T} = \pid{x^{2} - 2}$, as we aimed to prove.
	\end{proof}
	% section problem_1 (end)

	\section*{Problem 2}
	\label{sec:problem_2}
	List all polynomials $p(x)$ over $\Z_{2}$ that have degree 3, and determine which are reducible and which are irreducible.
	Write all polynomials as a product of their irreducible factors.
	For any irreducible, whether a degree 3 polynomial or a factor of something reducible, prove that it is irreducible.

	\begin{proof}
	Since $\Z_{2} = \{ 0,1 \}$, we have the following degree three polynomials from $\Z_{2}[x]$:

		\begin{figure}[h]
			\begin{tabular}{rlrl}
			$p_{1}(x) =$ & $x^{3} + x^{2} + x + 1$ & $p_{5}(x) =$ & $x^{3} + 1$ \\
			$p_{2}(x) =$ & $x^{3} + x + 1$ & $p_{6}(x) =$ & $x^{3} + x^{2}$ \\
			$p_{3}(x) =$ & $x^{3} + x^{2} + 1$ & $p_{7}(x) =$ & $x^{3} + x$ \\
			$p_{4}(x) =$ & $x^{3} + x^{2} + x$ & $p_{8}(x) =$ & $x^{3}$ 
			\end{tabular}
			\caption{Degree 3 polynomials from $\Z_{2}[x]$}
			\label{deg3polys}
		\end{figure}

	We will now determine which polynomials are irreducible, and for the reducible polynomials we write them as a product of their irreducible factors.
	Notice that since we are working in $\Z_{2}$, we have $R = \{ 0,1 \}$ as the only possible roots of our polynomials.
	Moreover, since we cannot use any theorems about factoring in $\Q$, we will be leaning on Theorem 8.22 quite extensively.
	Theorem 8.22 states that for a field $K$, and $a(x) \in K[x]$ with $\deg{a(x)} = 2$ or $\deg{a(x)} = 3$.
	The polynomial $a(x)$ is reducible over $K$ if and only if $a(x)$ has a root in $K$.
	Finally, recall that by the definition of irreducible and reducible, any linear factors are irreducible.

	\paragraph{$p_{1}(x)$}
	Let $p_{1}(x) = x^{3} + x^{2} + x + 1$.
	We compute the following,
		\begin{align*}
		p_{1}(0) &= 1 \neq 0, \\
		p_{1}(1) &= 4 \equiv 0 \mod{2}.
		\end{align*}
	Hence, $x = 1$ is a root, thus $p_{1}(x) = (x-1)b(x)$ for some $b(x) \in \Z_{2}[x]$; we determine $b(x)$ by polynomial long division: $$\polylongdiv{x^{3} + x^{2} + x + 1}{x - 1}$$
	Again, since we are operating in $\Z_{2}[x]$, we have $b(x) = x^{2} + 1$, and we can write $p_{1}(x) = (x + 1)(x^{2} + 1)$.
	We can notice that $b(1) = 0$, thus $1$ is also a root of $b(x)$, thus we can write $p_{1}(x) = (x-1)^{2}\tilde{b}(x)$ for some $\tilde{b}(x) \in \Z_{2}[x]$.
	Again, we use polynomial long division: $$\polylongdiv{x^{3} + x^{2} + x + 1}{(x - 1)^{2}}$$
	Thus, we have $p_{1}(x) = (x-1)^{2}(x+1)$, since these factors are linear, they are irreducible over $\Z_{2}$.
	Notice that we can also write $p_{1}(x) = (x+1)^{3}$ because $-1 \equiv 1 \mod{2}$.
	% paragraph p_ (end)
	\vspace{5mm}
	
	\paragraph{$p_{2}(x)$}
	Let $p_{2}(x) = x^{3} + x + 1$.
	We compute the following
		\begin{align*}
		p_{2}(0) &\equiv 1 \mod{2}, \\
		p_{2}(1) &\equiv 1 \mod{2}.
		\end{align*}
	Thus, $p_{2}(x)$ has no roots in $\Z_{2}$ and is irreducible over $\Z_{2}$.
	% paragraph p_ (end)
	\vspace{5mm}

	\paragraph{$p_{3}(x)$}
	Let $p_{3}(x) = x^{3} + x^{2} + 1$.
	We compute the following
		\begin{align*}
		p_{3}(0) &\equiv 1 \mod{2}, \\
		p_{3}(1) &\equiv 1 \mod{2}.
		\end{align*}
	Thus, $p_{3}(x)$ has no roots in $\Z_{2}$ and is irreducible over $\Z_{2}$.
	% paragraph p_ (end)
	\vspace{5mm}

	\paragraph{$p_{4}(x)$}
	Let $p_{4}(x) = x^{3}+x^{2}+x$.
	We can easily factor this as $p_{4}(x) = x(x^{2} + x + 1)$, since $x$ is linear, it remains to show that $\tilde{p}_{4}(x) = x^{2} + x + 1$ is either irreducible, or has reducible factors.
	We compute the following
		\begin{align*}
		\tilde{p}_{4}(0) &\equiv 1 \mod{2}, \\
		\tilde{p}_{4}(1) &\equiv 1 \mod{2}.
		\end{align*}
	Thus, $\tilde{p}_{4}(x)$ is irreducible over $\Z_{2}$, and we have $p_{4}$ as the following product of irreducible factors: $p_{4}(x) = x(x^{2} + x + 1)$
	% paragraph p_ (end)
	\vspace{5mm}

	\paragraph{$p_{5}(x)$}
	Let $p_{5}(x) = x^{3} + 1$.
	We compute the following
		\begin{align*}
		p_{5}(0) &\equiv 1 \mod{2}, \\
		p_{5}(1) &\equiv 0 \mod{2}.
		\end{align*}
	Hence 1 is a root and we have $p_{5}(x) = (x-1)b(x)$ for some $b(x) \in \Z_{2}[x]$.
	We compute $b(x)$: $$\polylongdiv{x^{3} + 1}{x - 1}$$
	Thus, $b(x) = x^{2} + x + 1$; at this point it should be fairly clear that $b(x)$ is irreducible given its only possible roots of 0 and 1, thus we have $p_{5}(x) = (x-1)(x^{2} + x + 1) \equiv (x+1)(x^{2} + x + 1) \mod{2}$.
	% paragraph p_ (end)
	\vspace{5mm}

	\paragraph{$p_{6}(x)$}
	Let $p_{6}(x) = x^{3} + x^{2} = x^{2}(x + 1)$, each factor is already linear, so we're done.
	% paragraph p_ (end)
	\vspace{5mm}

	\paragraph{$p_{7}(x)$}
	Let $p_{7}(x) = x^{3} + x = x(x^{2} + 1)$, since $x$ is linear, it remains to show that $x^{2} + 1$ is either irreducible or reducible.
	From previous work its clear that $x^{2} + 1 = (x+1)^{2}$, thus we have $p_{7}(x) = x(x+1)^{2}$.
	% paragraph p_ (end)
	\vspace{5mm}

	\paragraph{$p_{8}(x)$}
	Let $p_{8}(x) = x^{3}$.
	This is already a product of irreducible linear factors, so we're done.
	% paragraph p_ (end)
	\vspace{5mm}
	\end{proof}
	% section problem_2 (end)

	\section*{Problem 3}
	\label{sec:problem_3}
	Let $A$ and $B$ be commutative rings with unity, and let $f : A \to B$ be a ring homomorphism that is not identically zero.
	Prove that for all $b \in B$, there exists a ring homomorphism $F : A[x] \to B$ such that $F(a) = f(a)$ for all $a \in A$ and $F(x) = b$ (meaning that $F$ maps the polynomial $x$ to $b$).
	(Hint: use theorems; it's not necessary to prove everything from the definition).
	You may assume that $f(1_{A}) = 1_{B}$.

		\begin{lem}
		The composition of ring homomorphisms is a ring homomorphism.
		\end{lem}
		\begin{proof}
		Let $A,B,C$ be rings with ring homomorphisms $\phi: A \to B$ and $\psi: B \to C$.

		Additivity:
		\begin{align*}
		(\phi \circ \psi)(a + b) &= \phi(\psi(a+b)), \\
		&= \phi(\psi(a) + \psi(b)), \\
		&= \phi(\psi(a)) + \phi(\psi(b)), \\
		&= (\phi \circ \psi)(a) + (\phi \circ \psi)(b).
		\end{align*}

		Multiplicativity:
		\begin{align*}
		(\phi \circ \psi)(a \cdot b) &= \phi(\psi(a \cdot b)), \\
		&= \phi(\psi(a) \cdot \psi(b)), \\
		&= \phi(\psi(a)) \cdot \phi(\psi(b)), \\
		&= (\phi \circ \psi)(a) \cdot (\phi \circ \psi)(b).
		\end{align*}	
		\end{proof}

	\begin{proof}
	Recall the evaluation map $h_{c}: A[x] \to A$ defined by $h_{c}(a(x)) = a(c)$, where $c \in A$.
	By Theorem 7.35 we know that $h_{c}$ is a homomorphism.
	By our previous Lemma we know that the composition of ring homomorphisms is a ring homomorphism, so consider the ring homomorphism $(f \circ h_{c})$.

	Let $F = (f \circ h_{c})$, and let $a(x) \in A[x]$; recall that since $f$ is a function we know for some $a \in A$, there exists $b \in B$ such that $f(a) = b$.
	If $a(x) = a_{0}$, then for any $c \in A$ we have
		\begin{align*}
		F(a(x)) &= (f \circ h_{c})(a(x)), \\
		&= f(h_{c}(a_{0})), \\
		&= f(a_{0}), \\
		&= b.
		\end{align*}
	
	If $\deg{a(x)} > 0$, then we use $c = 0_{A}$ and get the following:
		\begin{align*}
		F(a(x)) &= (f \circ h_{0})(a(x)), \\
		&= f(h_{0}(a(x))), \\
		&= f(a(0)), \\
		&= f(a_{0}), \\
		&= b.
		\end{align*}
	\end{proof}

	Thus, the composition $(f \circ h_{c})$ is the ring homomorphism we were looking to find.
	% section problem_3 (end)
\end{document}