%!TEX output_directory = temp
\documentclass{amsbook}
	%% boilerplate packages
		\usepackage{amsmath}
		\usepackage{amssymb}
		\usepackage{amsthm}
		\usepackage[mathscr]{euscript}
		\usepackage{enumerate}
		\usepackage{graphicx}
		\usepackage{color}
	%% fatdot notation
		\makeatletter
			\newcommand*\fatdot{\mathpalette\fatdot@{.5}}
			\newcommand*\fatdot@[2]{\mathbin{\vcenter{\hbox{\scalebox{#2}{$\m@th#1\bullet$}}}}}
		\makeatother
	%% pretty letters
		\DeclareMathOperator{\ep}{\varepsilon}
		\DeclareMathOperator{\ph}{\varphi}
		\DeclareMathOperator{\Ell}{\mathscr{L}}
	%% congruences
		\renewcommand{\mod}[1]{\ (\mathrm{mod}\ #1)}
	%% brackets
		\newcommand{\pid}[1]{\langle #1 \rangle}
	%% order and characteristic
		\DeclareMathOperator{\ord}{\text{ord}}
	%% sets
		\DeclareMathOperator{\N}{\mathbb{N}}
		\DeclareMathOperator{\Z}{\mathbb{Z}}
		\DeclareMathOperator{\Zp}{\mathbb{Z}^{+}}
		\DeclareMathOperator{\Zm}{\mathbb{Z}^{-}}
		\DeclareMathOperator{\Q}{\mathbb{Q}}
		\DeclareMathOperator{\Qc}{\mathbb{Q}^{c}}
		\DeclareMathOperator{\R}{\mathbb{R}}
		\DeclareMathOperator{\C}{\mathbb{C}}
		%% powerset of a set
		\DeclareMathOperator{\pset}{\mathcal{P}}
		%% set of continuous functions in a certain variable
		\DeclareMathOperator{\cont}{\mathscr{C}}
		%% set of functions in a certain variable
		\DeclareMathOperator{\func}{\mathscr{F}}
\begin{document}
	\author{Alex Thies}
	\title{Self Study \\ Group Theory}

	\frontmatter

	\maketitle
	\newpage

	\tableofcontents
	\newpage

	\section*{Frequently Used Notation}
	\label{sec:frequently_used_notation}
	\begin{align*}
		f^{-1}(a)          & \hspace*{50pt} \text{the inverse image or preimage of $A$ under $f$.} \\
		a|b                & \hspace*{50pt} \text{$a$ divides $b$.} \\
		(a,b)              & \hspace*{50pt} \text{the greatest common divisor of $a,b$.} \\
		[a,b]              & \hspace*{50pt} \text{the least common multiple of $a,b$.} \\
		|A|                & \hspace*{50pt} \text{the number of elements in the set $A$, or the cardinality of $A$.} \\
		\Z,\Zp             & \hspace*{50pt} \text{the integers, the positive integers.} \\
		\Q,\Q^{+}          & \hspace*{50pt} \text{the rational numbers, the positive rational numbers.} \\
		\R,\R^{+}          & \hspace*{50pt} \text{the real numbers, the positive real numbers.} \\
		\C,\C^{\times}     & \hspace*{50pt} \text{the complex numbers, the nonzero complex numbers.} \\
		\Z_{n}             & \hspace*{50pt} \text{the integers modulo $n$.} \\
		A \times B         & \hspace*{50pt} \text{the cartesian product of $A$ and $B$.} \\
		D_{2n}             & \hspace*{50pt} \text{the dihedral group of order $2n$.} \\
		S_{n},S_{\Omega}   & \hspace*{50pt} \text{the symmetric group on $n$ letters, and on the set $\Omega$.} \\
		Q_{8}              & \hspace*{50pt} \text{the quaternion group of order 8.} \\
		\mathbb{F}_{N}     & \hspace*{50pt} \text{the finite field of $N$ elements.} \\
		GL_{n}(F),GL(V)    & \hspace*{50pt} \text{the general linear groups.} \\
		SL_{n}(F)          & \hspace*{50pt} \text{the special linear group.} \\
		M_{n}(F)           & \hspace*{50pt} \text{the set of all $n\times n$ matrices whose entries are elements of the set $F$.} \\
		A \cong B          & \hspace*{50pt} \text{$A$ is isomorphic to $B$.} \\
		\ker(f)            & \hspace*{50pt} \text{the kernel of the homomorphism $f$.} \\
		\ord(a)            & \hspace*{50pt} \text{the order of $a$.} \\
		f(A)               & \hspace*{50pt} \text{the image of the homomorphism $f$ over the group $A$.} \\
		H \triangleleft G  & \hspace*{50pt} \text{$H$ is a normal subgroup of $G$.} \\
		aH,Ha              & \hspace*{50pt} \text{the left coset, and right coset of $H$ with coset representative $a$.} \\
	\end{align*}

	Matrices will be capital bold letters, e.g., $\mathbf{X}, \mathbf{Y} \in GL(V)$, vectors will be lowercase bold letters, e.g., $\mathbf{u}, \mathbf{v} \in V$.
	% section frequently_used_notation (end)
	\newpage

	\chapter*{Preface}
	\label{sec:preface}
	Having utilized Chapters 4-7 from \textit{Essentials of Modern Algebra}, First Edition, by Cheryl Chute Miller in a course on Ring Theory (Math 391), I would like to gain an understanding of the Group Theory covered by Chapters 1-3.
	Additionally, I have decided to include some Group Theory material from \textit{Abstract Algebra}, Third Edition, by David Dummitt \& Richard Foote -- the textbook used in Math 444.
	Sections including material from the Math 444 textbook are preceded by a $\star$.
	% chapter preface (end)

	\mainmatter

	\chapter{Groups}
	\label{sec:groups}
		\section{Introduction to Groups}
		\label{sec:introduction_to_groups}
			\subsection*{Problem 1.17}
			\label{sub:problem_1_17}
			Write out the Cayley table for the group $(\Z_{6}, +_{6})$ and identify the inverse of each element.
			\begin{proof}[Solution] Figure \ref{117CT} is the Cayley table, observe that for $a \in \Z_{6}$ the inverse under addition modulo 6 is $a^{-1} = 6 - a$.
			\end{proof}
			% subsubsection problem_1_17 (end)

			\subsection*{Problem 1.18}
			\label{sub:problem_1_18}
			Find the 10 elements of the group $\Z_{5} \times \Z_{2}$, and write out the Cayley table. 
			Recall that its operation uses $+_{5}$ in the first coordinate and $+_{2}$ in the second. 
			Identify the inverse of each element.
			\begin{proof}[Solution] Figure \ref{118CT} is the Cayley table, observe that $\Z_{5} \times \Z_{2} = \{ (0,0), (1,0), (2,0),$ \\ 
			$(3,0), (4,0), (0,1), (1,1), (2,1), (3,1), (4,1) \}$.
			The identities are colored in red, thus for $(m,n) \in \Z_{5}\times\Z_{2}$ the inverses are $\left\{(m,n)^{-1} = \left( m^{-1} , n^{-1} \right) : m^{-1} = 5 - m \text{ and } \right.$\\ 
			$ \left. n^{-1} = 2 - n\right\}$.
			\end{proof}
			% subsubsection problem_1_18 (end)

			\subsection*{Problem 1.19}
			\label{sub:problem_1_19}
			What does Theorem 1.14 tell us about the entries in the Cayley table of a group?
			\begin{proof}[Solution]
			Theorem 1.14 states that: Suppose $G$ is a group under the operation $*$.
			\begin{enumerate}[(i)]
				\item There is exactly one element in $G$ that has the property of an identity.
				\item For each element $a \in G$ there is exactly one element of $G$ that satisfies the property of the inverse of $a$.
			\end{enumerate}
			
			My guess is that the identity elements $e_{G}$ are unique in their respective rows and columns of the table, like Sudoku.
			\end{proof}
			% subsubsection problem_1_19 (end)

			\subsection*{Problem 1.20}
			\label{sub:problem_1_20}
			Prove that a set with exactly one element, $A = \{a\}$, will always form a group since there is only one way to define an operation on the set. 
			Be sure to show that the properties of a group all hold.
			\begin{proof} Let $A$ be as above.
			Since the operation $*$ must be closed over $A$ the only operation that works is the identity, i.e., the only operation that holds over $A$ is $*:A \to A$ defined by $a \mapsto a$.
			We must show that $(A, *)$ is a group, i.e., we will show that $*$ is associative, that there exists a unique identity element $e_{G} \in A$, and that there exists a unique inverse element $a^{-1} \in A$ for each $a \in A$.

			Notice that $(a*a)*a = a * a = a$, and that $a*(a*a) = a * a = a$, thus by transitvity we have that $*$ is associative.
			Additionally, note tha $a \in A$ satisfies the necessary and sufficient properties of both the identity $e_{G}$, and the inverse (of itself) $a^{-1} = a$.
			Finally, notice that $*$ is obviously commutative, therefore the group formed from a set containing only one element is always an abelian group.
			\end{proof}
			% subsubsection problem_1_20 (end)

			\subsection*{Problem 1.21}
			\label{sub:problem_1_21}
			We can even create groups with games! 
			Consider four cups placed in a square pattern on a table. 
			If we have a penny in one of the cups there are four ways we can move it to another cup: Horizontally, Vertically, Diagonally, or Stay where it is. 
			We will label them $H, V, D, S$. 
			To define an operation, consider two movements in a row, i.e., $x*y$ means we first move the penny as $x$ tells us to, then after that move the penny as $y$ instructs. 
			For example, $H * V = D$ since if we first move it horizontally and then vertically altogether we have moved it diagonally. 
			Create the Cayley table for this group, identify the identity, and the inverse of each element.
			\begin{proof}[Solution] Call this group $(G, *)$.
			Figure \ref{121CT} is the Cayley table, note that $S$ is the identity and that for each $a \in G$, we have the inverse $a^{-1} = a$.
			Notice that groups with this property (that each element is its own inverse) have symmetric Cayley tables.
			\end{proof}
			% subsubsection problem_1_21 (end)

			\subsection*{Problem 1.22}
			\label{sub:problem_1_22}
			Prove that the set $A = \left\{ \begin{pmatrix} 1 & 0 \\ 0 & -1 \end{pmatrix}, \begin{pmatrix} 1 & 0 \\ 0 & 1 \end{pmatrix}, \begin{pmatrix} -1 & 0 \\ 0 & 1 \end{pmatrix}, \right.$ \\
			$ \left. \begin{pmatrix} -1 & 0 \\ 0 & -1 \end{pmatrix} \right\}$ is a group under matrix multiplication. 
			Is it abelian?
			\begin{proof} Recall that matrix multiplication is always associative, it remains to find identities and inverses.
			Figure \ref{122CT} is the Cayley table, note that the two-by-two identity matrix contained in $A$ is the identity of $(A,\times)$, i.e., $e_{G} = \mathbf{I}_{2}$ and that each element of $A$ is its own inverse under matrix multiplication.
			Finally, notice that Figure \ref{122CT} is symmetric, therefore $(A, \times)$ is an abelian group.
			\end{proof}
			% subsubsection problem_1_22 (end)

			\subsection*{Problem 1.23}
			\label{sub:problem_1_23}
			Verify that $\func(\R)$ is a group using the addition of functions as defined in Example 1.13.
			\begin{proof}[Solution] We will show that $(\func(\R), +)$ is a group under normal function addition.
			We will do this by showing that $+$ is associative, and that an identity and inverses exist under $+$.

			Let $f,g,h \in \func(\R)$ and compute that $f+(g+h) = f(x) + (g + h)(x) = f(x) + g(x) + h(x) = (f + g)(x) +h(x) = (f + g) + h$, thus $+$ is associative over $\func(\R)$.
			Notice that $0(x) \in \func(\R)$ is the identity, thus for each $f \in \func(\R)$, we have $-f \in \func(\R)$ as an inverse.
			Hence, $(\func(\R), +)$ is a group.
			\end{proof}
			% subsubsection problem_1_23 (end)
		% section introduction_to_groups (end)

		\section{Basic Algebra in Groups}
		\label{sec:basic_algebra_in_groups}
			\subsection*{Problem 1.24}
			\label{sub:problem_1_24}
			Prove the Cancellation Law (Theorem 1.20). 
			Do not use any theorems that occur after it in the text. \\

			\noindent\textit{Theorem}. (Cancellation Law) Suppose $G$ is a group and $a,b,c \in G$.
			\begin{enumerate}[(i)]
				\item If $ab = ac$ then $b = c$.
				\item If $ba = ca$ then $b = c$.
			\end{enumerate}
			
			\begin{proof} Let $G, a,b,c$ be as above.
			Recall that since $G$ is a group, there exists a unique inverse $a^{-1} \in G$ such that $aa^{-1} = e_{G} = a^{-1}a$.
			Suppose $ab = ac$, we compute the following:
				\begin{align*}
				ab &= ac, \\
				a^{-1}ab &= a^{-1}ac, \\
				e_{G}b &= e_{G}c, \\
				b &= c.
				\end{align*}
			Thus $ab = ac$ implies that $b=c$ as desired.
			A similar argument shows the second part of the theorem.
				\begin{align*}
				ba &= ca, \\
				baa^{-1} &= caa^{-1}, \\
				be_{G} &= ce_{G}, \\
				b &=  c.
				\end{align*}
			Thus $ba = ca$ implies that $b=c$, and the theorem is proved.
			\end{proof}
			% subsubsection problem_1_24 (end)

			\subsection*{Problem 1.25}
			\label{sub:problem_1_25}
			Assume $G$ is a group and $a \in G$. 
			Consider a fixed integer $k$ and use PMI (Theorem 0.3) to prove that for all $n \in \N, a^{n}a^{k} = a^{n+k}$. 
			You will need to consider cases here since $k$ can be positive, negative, or 0.
			\begin{proof}
			\end{proof}
			% subsubsection problem_1_25 (end)

			\subsection*{Problem 1.26}
			\label{sub:problem_1_26}
			Prove Theorem 1.23 (ii) for the case of $n > 0$ and $m < 0$.
			\begin{proof}
			\end{proof}
			% subsubsection problem_1_26 (end)

			\subsection*{Problem 1.27}
			\label{sub:problem_1_27}
			Assume $G$ is a group and $a \in G$. 
			Prove by PMI (Theorem 0.3) that for $n \in \N, \left(a^{n}\right)^{-1} = a^{-n}$.
			\begin{proof} Let $a,G$ be as above.
			Notice that for $n=1$ we have $(a^{1})^{-1} = a^{-1}$.
			Assume that for some natural number $n$ we have $(a^{n})^{-1} = a^{-n}$.
			We compute the following,
				\begin{align*}
					\left( a^{n+1} \right)^{-1} &= \left( a^{n} a \right)^{-1}, \\
					&= (a^{n})^{-1} a^{-1}, \\
					&= a^{-n} a^{-1}, \\
					&= a^{-n-1}, \\
					&= a^{-(n+1)}.
				\end{align*}
			Thus, for the given proposition $P$ we have shown that $P(n) \Rightarrow P(n+1)$.			
			\end{proof}
			% subsubsection problem_1_27 (end)

			\subsection*{Problem 1.28}
			\label{sub:problem_1_28}
			Suppose $G$ is a group with $a, b \in G$. 
			Prove: If $a^{3} = b$, then $b = aba^{-1}$.
			\begin{proof} Let $G,a,b$ be as above and suppose that $a^{3} = b$, we compute the following,
				\begin{align*}
					a^{3} &= b, \\
					a(aaa)a^{-1} &= aba^{-1}, \\
					(aaa)\left(aa^{-1}\right) &= aba^{-1}, \\
					a^{3}e_{G} &= aba^{-1}, \\
					b &= aba^{-1}.
				\end{align*}
			\end{proof}
			% subsubsection problem_1_28 (end)

			\subsection*{Problem 1.29}
			\label{sub:problem_1_29}
			Suppose $G$ is a group and $a, b \in G$. 
			Prove by PMI for all $n \in \N, (aba^{-1})^{n} = ab^{n}a^{-1}$.
			Do not assume $G$ is abelian.
			\begin{proof} Let $G,a,b,n$ be as above.
			We proceed by mathematical induction over the natural numbers $n \in \N$.
			Notice the base case $n = 1$ holds since $$\left(aba^{-1}\right)^{1} = ab^{1}a^{-1}.$$
			Thus, assume that for some $n \in \N$ the following is true $$(aba^{-1})^{n} = ab^{n}a^{-1}.$$
			We compute the following,
				\begin{align*}
					\left(aba^{-1}\right)^{n+1} &= \left(aba^{-1}\right)\left(aba^{-1}\right)^{n} = \left(aba^{-1}\right)^{n}\left(aba^{-1}\right),\\
					&= \left(aba^{-1}\right) ab^{n}a^{-1} = ab^{n}a^{-1}\left(aba^{-1}\right), \\
					&= ab\left(a^{-1}a\right)b^{n}a^{-1} = ab^{n}\left(a^{-1}a\right)ba^{-1}, \\
					&= ab(e_{G})b^{n}a^{-1} = ab(e_{G})b^{n}a^{-1}, \\
					&= abb^{n}a^{-1}, \\
					&= ab^{n+1}a^{-1}.
				\end{align*}
			Hence, for the given proposition $P$, we have shown that $P(n) \Rightarrow P(n+1)$.			
			\end{proof}
			% subsubsection problem_1_29 (end)

			\subsection*{Problem 1.30}
			\label{sub:problem_1_30}
			Suppose $G$ is a group and $a, b \in G$. 
			Prove: If $n \in \N$ and $\ord(b) = n$, then $\ord(aba^{-1}) = n$.
			Use the result of the previous exercise to help.
			\begin{proof} Let $G,a,b$ be as above, suppose $n \in \N$ and $\ord(a) = n$.
			Recall that for all $n \in \N$ we have $(aba^{-1})^{n} = ab^{n}a^{-1}$, we compute that
				\begin{align*}
					(aba^{-1})^{n} &= ab^{n}a^{-1}, \\
					&= a e_{G} a^{-1}, \\
					&= aa^{-1}, \\
					&= e_{G}.
				\end{align*}
			Thus, $\ord(b) = n$ implies that $\ord(aba^{-1}) = n$.
			\end{proof}
			% subsubsection problem_1_30 (end)

			\subsection*{Problem 1.31}
			\label{sub:problem_1_31}
			Find the order of each element in the group $(\Z_{12}, +_{12})$.
			\begin{proof}[Solution] We compute the following,\\

				\begin{tabular}{cc|cc|cc}
				$n$ & $\ord(n)$ & $n$ & $\ord(n)$ & $n$ & $\ord(n)$ \\
				\hline
				\hline
				0 & 0  & 4 & 2  & 8  & 2 \\
				1 & 11 & 5 & 11 & 9  & 3 \\
				2 & 5  & 6 & 1  & 10 & 5 \\
				3 & 3  & 7 & 11 & 11 & 11
				\end{tabular}
			\end{proof}
			% subsubsection problem_1_31 (end)

			\subsection*{Problem 1.32}
			\label{sub:problem_1_32}
			Find the order of each element in the group $(\Z_{7} +_{7})$.
			\begin{proof}[Solution] We compute that for each $a \in \Z_{7}/\{0\}$ we have $\ord(a) = 6$ and $\ord(0)=0$.
			\end{proof}
			% subsubsection problem_1_4_32 (end)

			\subsection*{Problem 1.33}
			\label{sub:problem_1_4_33}
			Find the order of each element in the group $A = \left\{ \begin{pmatrix} 1 & 0 \\ 0 & -1 \end{pmatrix}, \right.$ \\
			$\left. \begin{pmatrix} 1 & 0 \\ 0 & 1 \end{pmatrix}, \begin{pmatrix} -1 & 0 \\ 0 & 1 \end{pmatrix}, \begin{pmatrix} -1 & 0 \\ 0 & -1 \end{pmatrix} \right\}$ under matrix multiplication.
			\begin{proof}[Solution] Note that $e_{A} = \mathbf{I}_{2}$, we compute that $\ord(e_{A}) = 0$, and that for each other $a \in A$ we have $\ord(a) = 1$.
			\end{proof}
			% subsubsection problem_1_33 (end)

			\subsection*{Problem 1.34}
			\label{sub:problem_1_34}
			Find the order of each element in the group $G = \{Horizontally, $ \\
			$Vertically, Diagonally, Stay\}$ from exercise 21 above (use the operation defined in that exercise).
			\begin{proof}[Solution] Recall that $e_{G} = S$, we compute that $\ord(e_{A}) = 0$, and that for each other $a \in A$ we have $\ord(a) = 1$.
			\end{proof}
			% subsubsection problem_1_34 (end)

			\subsection*{Problem 1.35}
			\label{sub:problem_1_35}
			Complete the proof of Theorem 1.26.\\

			\noindent\textit{Theorem}. Suppose $G$ is a group and $a \in G$ with $\ord(a) = n$ for some $n \in \N$.
			For any $t \in \Z$, $a^{t} = e_{G}$ if and only if $n$ divides $t$, i.e., $t = nq$ for some unique integer $q$.
			\begin{proof} $\Rightarrow)$ was done in the book.

			$\Leftarrow)$ Let $G,a,n$ be as above.
			Let $t \in \Z$ such that $n|t$.
			We will show that $a^{t} = e_{G}$ for all $t$ such that $n|t$.
				\begin{align*}
					a^{t} &= a^{nq}, \\
					&= \left( a^{n} \right)^{q}, \\
					&= e_{G}^{n}, \\
					&= e_{G}.
				\end{align*}
			\end{proof}
			% subsubsection problem_1_35 (end)

			\subsection*{Problem 1.36}
			\label{sub:problem_1_36}
			Suppose $G$ is a group and $a \in G$. 
			Prove: if $\ord(a) = 6$ then $\ord(a^{5}) = 6$. 
			Do any other elements $a^{2}, a^{3}, a^{4}$ have have order 6? 
			Explain.
			\begin{proof}
			\end{proof}
			% subsubsection problem_1_36 (end)

			\subsection*{Problem 1.37}
			\label{sub:problem_1_37}
			Suppose $G$ is a group and $a \in G$. 
			Assume $a^{50} = e_{G}$ but $a^{75} \neq e_{G}$ and $a^{10} \neq e_{G}$. 
			Find the order of $a$, and prove that your answer is correct.
			\begin{proof}[Solution] Let $G,a$ be as above.
			We claim that $\ord(a) = 50$, but suppose that there exists a different $m \in \Z$ such that $m \leq 50$ and $a^{m} = e_{G}$.
			We will show that $m$ must be 50.

			We know by a theorem that if there exist $t \in \Z$ such that $a^{t} = e_{G}$, and $\ord(a) = n$, then $n|t$.
			Thus, since $a^{50} = e_{G}$ we know that $m|50$; therefore we can claim that $m \in \{ 1,2,5,10,25,50 \}$.
			We also know that $a^{75} \neq e_{G}$, thus the divisors of 75 cannot be candidates for $m$, thus we can narrow down the set of possible $m$ to $\{ 2,10,50 \}$.
			Finally, we know that $a^{10} \neq e_{G}$, by the above argument we know that $m$ can neither be 2 nor 10, thus $m = 50$ and $\ord(a) = 50$ as we aimed to show.
			\end{proof}
			% subsubsection problem_1_37 (end)

			\subsection*{Problem 1.38}
			\label{sub:problem_1_38}
			Suppose $G$ is a group and $a \in G$ with $a \neq e_{G}$.
			\textbf{Prove}: If $a^{p}=e_{G}$ for some prime number $p$, then $\ord(a) = p$.
			\begin{proof}
			Let $G,a$ be as above.
			Suppose $a^{p} = e_{G}$ for some prime number $p$.
			Further, suppose that there exists $n \in \Z$, $n \leq p$ such that $a^{n} = e_{G}$; we will show that $n = p$.

			Since $a^{p} = e_{G}$, $a^{n} = e_{G}$, and $n \leq p$ we know that $n|p$.
			But since $p$ is prime, that means that $n=1$, or $n = p$.
			\begin{enumerate}
				\item[\textbf{Case 1}:] Let $n = p$, we're done.
				\item[\textbf{Case 2}:] Let $n = 1$, so we have $a^{1} = e_{G}$, but we assumed $a \neq a_{G}$, so this case fails.
			\end{enumerate}
			Thus, $\ord(a) = p$.		
			\end{proof}
			% subsubsection problem_1_38 (end)

			\subsection*{Problem 1.39}
			\label{sub:problem_1_39}
			Suppose $G$ is a group and $a \in G$. 
			Prove: If $\ord(a)$ is an even integer then for any odd $k \in \Z$, $a^{k} \neq e_{G}$. 
			Would the statement still be true if the words even and odd changed places, i.e., if $\ord(a)$ is odd then for any even $k \in \Z, a^{k} \neq e_{G}$? 
			Give a brief justification of your answer.
			\begin{proof}
			Let $a,G$ be as above.
			Suppose that $\ord(a) = 2n$ for some $n \in \Z$, and $k \in \Z$ such that $k = 2m+1$ for some $m \in \Z$.
			Then we can compute $a^{2n} = e_{G} \iff a^{2n+1} = a$.
			Thus if $a$ has even order, $a$ raised to an odd power can never equal the identity.
			A similar argument shows the same for $a$ with odd order and even powers.
			\end{proof}
			% subsubsection problem_1_39 (end)
		% section basic_algebra_in_groups (end)

		\section{Subgroups}
		\label{sec:subgroups}
			\subsection*{Problem 1.40}
			\label{sub:problem_1_40}
			Prove or disprove that $H = \{ 0,3, 6, 9 \}$ is a subgroup of $(\Z_{12}, +_{12})$.
			\begin{proof} Its obvious that $H$ is nonempty, observe Figure \ref{140CT} to see that $H$ is closed under $+_{12}$, has an identity, and each element of $H$ has an inverse under $+_{12}$, thus $H$ is a subgroup of $(\Z_{12}, +_{12})$.
			\end{proof}
			% subsubsection problem_1_40 (end)

			\subsection*{Problem 1.41}
			\label{sub:problem_1_41}
			Prove or disprove that $H = \{ 0,3, 6, 9 \}$ is a subgroup of $(\Z_{10}, +_{10})$.
			\begin{proof} Notice that $3 +_{10} 9 = 2$ and $2 \notin H$ thus $H$ is not closed under $+_{10}$ and not a subgroup of $\Z_{10}$
			\end{proof}
			% subsubsection problem_1_41 (end)

			\subsection*{Problem 1.42}
			\label{sub:problem_1_42}
			Prove that $H = \{ 0,2, 4, 8, 16 \}$ is a subgroup of $(\Z_{18}, +_{18})$.
			\begin{proof} Observe rows 3,4 and columns 3,4 from Figure \ref{142CT} indicate that there does not exist $8^{-1} \in H$ under $+_{18}$, thus $H$ is not a subgroup of $\Z_{18}$.
			\end{proof}
			% subsubsection problem_1_42 (end)

			\subsection*{Problem 1.43}
			\label{sub:problem_1_43}
			Prove or disprove that $H = \{(0, 0), (0, 2), (1, 0), (1,2) \}$ is a subgroup of $\Z_{2} \times \Z_{4}$ under the operation using $+_{2}$ in the first coordinate and $+_{4}$ in the second coordinate.
			\begin{proof} As in Problem 1.40 it is obvious to see that $H$ is nonempty, observe Figure \ref{143CT} to see that $H$ has all of the properties of a subgroup of $\Z_{2} \times \Z_{4}$ under the operation using $+_{2}$ in the first coordinate and $+_{4}$ in the second coordinate.
			\end{proof}
			% subsubsection problem_1_43 (end)

			\subsection*{Problem 1.44}
			\label{sub:problem_1_44}
			Prove or disprove that $H = \{ 0,2, 4, 6 \}$ is a subgroup of $(\Z_{7} +_{7})$.
			\begin{proof} Notice that $2 +_{7} 6 = 1$ and that $1 \notin H$, thus $H$ is not closed under $+_{7}$ and not a subgroup of $\Z_{7}$.
			\end{proof}
			% subsubsection problem_1_44 (end)

			\subsection*{Problem 1.45}
			\label{sub:problem_1_45}
			Determine if the set $G = \{ \frac{n}{3} : n \in \Z \}$ is a subgroup of $(\Q,+)$. 
			Either prove that it is or give a specific example of how it fails.
			\begin{proof} It is obvious to see that $G$ is nonempty, it remains to show that $G$ is closed under addition, has an inverse, and that each element of $G$ has an inverse under addition.

			Since integer addition is closed, we know that $G$ is closed under addition.
			As before, since integer addition has an identity and inverse, we know that $G$ has the same properties, thus $G$ satisfies all of the necessary and sufficient conditions required of a subgroup.
			\end{proof}
			% subsubsection problem_1_45 (end)

			\subsection*{Problem 1.46}
			\label{sub:problem_1_46}
			Determine if the set $\left\{ \frac{1}{n} : n \in \Z, n \neq 0 \right\}$ is a subgroup of $(\Q^{*},\cdot)$ (nonzero rational numbers). 
			Either prove that it is or give a specific example of how it fails.
			\begin{proof} Notice that there do not exist inverses under $\cdot$ in $G$ since the numerator is fixed and $n \in \Z/\{0\}$, thus $G$ is not a subgroup of $(\Q^{*},\cdot)$.
			\end{proof}
			% subsubsection problem_1_46 (end)

			\subsection*{Problem 1.47}
			\label{sub:problem_1_47}
			Determine if $K = \{ 3n : n \in \Z \}$ is a subgroup of $(\Z,+)$. 
			Either prove that it is or give a specific example of how it fails.
			\begin{proof} Notice that $K$ is nonempty, we will now show that $K$ is closed under addition, and contains inverses under addition.

			Let $a,b \in K$, then 
				\begin{align*}
					a+b &= 3n + 3m, \\
					&= 3(n + m), \\
					&= 3\ell \ \ \ \text{for some integer $\ell = n+m$.}
				\end{align*}
			Thus $K$ is closed under addition.
			Since $-n \in \Z$ we have $a^{-1} = 3(-n) \in K$, so for each $a \in K$ there exists a unique $a^{-1} \in K$.
			Hence, $K$ is a subgroup.	
			\end{proof}
			% subsubsection problem_1_47 (end)

			\subsection*{Problem 1.48}
			\label{sub:problem_1_48}
			Determine if $K = \{ 3 + n : n \in \Z \}$ is a subgroup of $(\Z, +)$.
			Either prove that it is or give a specific example of how it fails.
			\begin{proof}
			\end{proof}
			% subsubsection problem_1_48 (end)

			\subsection*{Problem 1.49}
			\label{sub:problem_1_49}
			Consider the set $H = \left\{ \begin{pmatrix} a & b \\ -b & -a \end{pmatrix} : a,b \in \R \right\}$.
			Determine if $H$ is a subgroup of $M_{2}(\R)$ under usual addition of matrices.
			\begin{proof} Let $H$ be as above, notice that $H \neq \emptyset$ is obvious.
			We will show that $H$ is closed under matrix addition.
			Let $\mathbf{X},\mathbf{Y} \in H$, then
				\begin{align*}
					\mathbf{X} + \mathbf{Y} &= \begin{pmatrix} a & b \\ -b & -a \end{pmatrix} + \begin{pmatrix} c & d \\ -d & -c \end{pmatrix}, \\
					&= \begin{pmatrix} a+c & b+d \\ -(b+d) & -(a+c) \end{pmatrix}.
				\end{align*}
			Since $\R$ is closed under addition we know that $\mathbf{X}+\mathbf{Y} \in H$, thus $H$ is closed under matrix addition.
			It remains to show that $H$ has inverses for each of its elements.
			Notice that for $\mathbf{X} \in H$ we have $\mathbf{X}^{-1} = \begin{pmatrix} -a & -b \\ b & a \end{pmatrix}$ such that $a \neq b$.
			Hence, $H$ is a subgroup of $M_{2}(\R)$.
			\end{proof}
			% subsubsection problem_1_49 (end)

			\subsection*{Problem 1.50}
			\label{sub:problem_1_50}
			Determine if the set $K = \{ x^{2} : x \in \R \}$ is a subgroup of $(\R, +)$. 
			(Note that $x^{2}$ means usual multiplication.)
			\begin{proof}
			Notice that $x^{2} + y^{2} \neq (x + y)^{2}$ therefore $K$ is not closed under addition and not a subgroup.
			\end{proof}
			% subsubsection problem_1_50 (end)

			\subsection*{Problem 1.51}
			\label{sub:problem_1_51}
			Determine if the set $K = \{ x^{2} : x \in \R^{+} \}$ is a subgroup of $(\R^{+},\cdot)$. 
			(Note that $x^{2}$ means usual multiplication, and $\R^{+}$ is the set of positive real numbers.)
			\begin{proof} Let $K$ be as above, notice that $K \neq \emptyset$ is obvious.
			We will show that $K$ is closed under $\cdot$ and has inverses for each of its elements under $\cdot$.
			Let $a,b \in K$, then there exist $x,y \in \R$ such that $a = x^{2}, b = y^{2}$.
			So we have $$ab = x^{2}y^{2} = (xy)^{2}$$ and since $xy \in \R$ we have that $K$ is closed.
			It remains to show that $K$ has inverses for each of its elements under $\cdot$.
			Notice that $a^{-1} = x^{-2} = 1/x^{2}$ and that $1/x^{2} \in \R^{+}$.
			Hence, $K$ is a subgroup of $(\R^{+},\cdot)$.
			\end{proof}
			% subsubsection problem_1_51 (end)

			\subsection*{Problem 1.52}
			\label{sub:problem_1_52}
			Let $G$ be a group, and define the set $H = \{ a \in G : a^{2} = e_{G} \}$. 
			Prove: If $G$ is abelian then $H$ is a subgroup of $G$. 
			Where does your proof break down if we do not know $G$ is abelian?
			\begin{proof} Let $G,H$ be as above and suppose $G$ is an abelian group.
			Consider $a \in G$, since $G$ is abelian we know that $a^{2} \in G$ so $H$ is nonempty.
			We will now show that $H$ is closed under $*$.
			Let $\alpha,\beta \in H$, then for $a,b \in G$ we have $\alpha = a^{2}, \beta = b^{2}$, thus,
				\begin{align*}
					\alpha * \beta &= a^{2} * b^{2}, \\
					&= a(ab)b, \\
					&= a(ba)b, \ \ \ \ \text{this is where we need that $G$ is abelian;} \\
					&= (ab)^{2}.
				\end{align*}
			Thus, $\alpha\beta \in H$, it remains to show that for each $\alpha \in H$ we have $\alpha^{-1} \in H$ such that $\alpha \alpha^{-1} = e_{H}$.
			We can see that $\alpha^{-1} = a^{-2}$ fairly easily.
			Thus, we have shown that the subset of squares of a group is a subgroup.
			\end{proof}
			% subsubsection problem_1_52 (end)

			\subsection*{Problem 1.53}
			\label{sub:problem_1_53}
			Prove Theorem 1.33.

			\noindent{\textit{Lemma}:} Let $G$ be a group, the identity of $G$ is also the identity for $H$, any subgroup of $G$ i.e., $e_{G} = e_{H}$.
			\begin{proof}
			\end{proof}		
			\noindent{\textit{Corollary}:} Any pair of subgroups of $G$ are not disjoint because they have their shared identity element in common.\\

			\noindent{\textit{Theorem}.} Let $G$ be a group. 
			If $H_{1}$ and $H_{1}$ are both subgroups of $G$ then $H_{1} \cap H_{2}$ is also a subgroup of $G$.
			\begin{proof} Let $G,H_{1},H_{2}$ be as above.
			We must show that $H_{1} \cap H_{2}$ is nonempty, closed under $*$ and has an inverse for each of its elements under $*$, where $*$ is from $(G,*)$.
			Since $H_{i}$ are subgroups they're nonempty, but we must show that their intersection is also nonempty, i.e., we must show that any pair of subgroups of $G$ are not disjoint, which we did in the Lemma.
			Thus, since $H_{1},H_{2}$ are not disjoint, $H_{1} \cap H_{2}$ is nonempty; it remains to show that $H_{1} \cap H_{2}$ is closed under $*$ and has inverses for each of its elements under $*$.

			Let $a,b \in H_{1} \cap H_{2}$, thus $a,b \in H_{1},H_{2},G$ as well.
			Notice that $ab \in H_{1},H_{2}$ as well since $H_{1},H_{2}$ are subgroups of $G$, hence $ab \in H_{1} \cap H_{2}$ and thus $H_{1} \cap H_{2}$ is closed under $*$.

			Since $G,H_{1},H_{2}$ are groups, for each $a$ in these groups we have $a^{-1}$, thus $a_{-1} \in H_{1} \cap H_{2}$.
			Hence, $H_{1} \cap H_{2}$ is a subgroup of $G$.
			\end{proof}
			% subsubsection problem_1_53 (end)
		% section subgroups (end)

		\section{Homomorphisms}
		\label{sec:homomorphisms}
			\subsection*{Problem 1.54}
			\label{sub:problem_1_54}
			Determine if the function $f : \Z_{8} \to \Z_{4}$ defined by $f(x) = 2x \mod{4}$ is a homomorphism between groups $(\Z_{8}, +_{8})$ and $(\Z_{4}, +_{4})$ In this definition $2x$ means usual integer multiplication.
			\begin{proof} Recall that since $\Z_{8}$ and $\Z_{4}$ are groups, they are closed.
			Notice from Figure \ref{154f1} that the only ouputs of $f$ over $\Z_{8}$ are 0 and 2.
			Additionally, observe from Figure \ref{154CT} that the only outputs of $+_{4}$ given inputs 0 and 2 are again 0 and 2, thus $f$ is a homomorphism.
			Thus, for each $a,b \in \Z_{8}$ we have $f(a)f(b)=f(ab)$.
			\end{proof}
			% subsubsection problem_1_54 (end)

			\subsection*{Problem 1.55}
			\label{sub:problem_1_55}
			Prove that for any $n \in \N$ with $n > 1$, the function $f:\Z \to \Z_{n}$ defined by $f(x) = x \mod{n}$ is a homomorphism between groups $(\Z,+)$ and $(\Z_{n},+_{n})$.
			\begin{proof} Let $a,b \in \Z$.
			Observe that $f(a) = a \mod{n}$ and $f(b) = b \mod{n}$, thus their sum is $f(a)+f(b) = a + b \mod{n}$.
			Also, we have $f(a + b) = a + b \mod{n}$, so $f(a+b) = f(a) + f(b)$ as desired.
			\end{proof}
			% subsubsection problem_1_55(end)

			\subsection*{Problem 1.56}
			\label{sub:problem_1_56}
			Prove that the function $f: \Z \to \Z$ defined by $f(x) = 4x$ is a homomorphism of $(\Z, +)$. 
			(Note that $4x$ means usual integer multiplication.)
			\begin{proof} Let $a,b \in \Z$ and $f$ be as above.
			Then $f(a) = 4a$, $f(b) = 4b$, and $f(a+b) = 4(a+b)$.
			Additionally, we have $f(a)+f(b) = 4(a+b)$.
			Hence, $f$ is a homomorphism.
			\end{proof}
			% subsubsection problem_1_56 (end)

			\subsection*{Problem 1.57}
			\label{sub:problem_1_57}
			Determine if the function $g : \R \to \R$ defined by $g(x) = x^{2} + 1$ is a homomorphism of$(\R, +)$.
			\begin{proof} Let $g$ be as above, suppose $a,b \in \R$, then $g(a+b) = (a+b)^{2} + 1 = a^{2} + 2ab + b^{2} + 1$.
			Additionally, we have $g(a) + g(b) = a^{2} + b^{2} + 2$.
			Notice that $g(a) + g(b) \neq g(a+b)$, hence $g$ is \underline{not} a homomorphism.
			\end{proof}
			% subsubsection problem_1_57 (end)

			\subsection*{Problem 1.58}
			\label{sub:problem_1_58}
			Determine if the function $h : \Z \to \Z$ defined by $h(x) = x^{3}$ is a homomorphism of $(\Z, +)$.
			\begin{proof}
			\end{proof}
			% subsubsection problem_1_58 (end)

			\subsection*{Problem 1.59}
			\label{sub:problem_1_59}
			Determine if the function $g : \R^{+} \to \R^{+}$ defined by $g(x) = 1/x$ is a homomorphism of $(\R^{+},\cdot)$. 
			Recall that $\R^{+}$ is the set of positive real numbers.
			\begin{proof}
			\end{proof}
			% subsubsection problem_1_59 (end)

			\subsection*{Problem 1.60}
			\label{sub:problem_1_60}
			Determine if the function $f : M_{2}(\R) \to \R$ defined by $f\left( \begin{bmatrix} a & b \\ c & d \end{bmatrix} \right) = a + b$ is a homomorphism. 
			Recall that both $M_{2}(\R)$ and $\R$ use an operation of addition.
			\begin{proof}
			\end{proof}
			% subsubsection problem_1_60 (end)

			\subsection*{Problem 1.61}
			\label{sub:problem_1_61}
			Use PMI to prove (iii) of Theorem 1.37.\\

			\noindent{\textit{Theorem}.} Let $G$ and $K$ be groups and suppose $f: G \to K$ is a homomorphism. 
			Then the following must hold:
			\begin{enumerate}[(i)]
				\item $f(e_{G}) = e_{K}$
				\item $f\left( a^{-1} \right) = \left( f(a) \right)^{-1}$ for each $a \in G$.
				\item For every $n \in \Z$ and $a \in G$, $\left( f(a) \right)^{n} = f\left( a^{n} \right)$.
				\item For every $a \in G$, if $a$ has finite order then $\ord(f(a))$ evenly divides $\ord(a)$.
			\end{enumerate}
			
			\begin{proof} Let $G,K,a,n,f$ be as above.
			Notice that for $n=1$ we have $f(a)^{1} = f(a^{1})$.
			Assume that for some $n \in \N$ that $\left( f(a) \right)^{n} = f\left( a^{n} \right)$.
			We compute that,
				\begin{align*}
					\left( f(a) \right)^{n+1} &= \left( f(a) \right)^{n}f(a), \\
					&= f\left(a^{n}\right)f\left(a\right), \\
					&= f\left(a^{n}a\right), \\
					&= f\left(a^{n+1}\right).
				\end{align*}
			Thus we have shown that for the given proposition $P$, that $P(n) \Rightarrow P(n+1)$.			
			\end{proof}
			% subsubsection problem_1_61 (end)

			\subsection*{Problem 1.62}
			\label{sub:problem_1_62}
			Prove (iv) of Theorem 1.37.
			\begin{proof} Let $G,a,f$ be as in the previous problem.
			Since $a$ has finite order, there exists an $n \in \N$ such that $a^{n} = e_{G}$, and $n$ is the least integer with this property.
			Consider $f(a^{n}) = f(e_{G}) = e_{K}$.
			From the previous problem we also know that $\left( f(a) \right)^{n} = f\left( a^{n} \right)$, thus $\left( f(a) \right)^{n} = e_{K}$, therefore $n$ is either the order of $f(a)$, or the order of $f(a) = m$ such that $m|n$.
			Thus, $\ord(f(a))|\ord(a)$.
			\end{proof}
			% subsubsection problem_1_62 (end)

			\subsection*{Problem 1.63}
			\label{sub:problem_1_63}
			Prove (i) of Theorem 1.40.\\

			\noindent{\textit{Theorem}.} Suppose $G$ and $K$ are groups and $f : G \to K$ is an isomorphism.
			\begin{enumerate}[(i)]
				\item $G$ is abelian if and only if $K$ is abelian.
				\item For every $a \in G$, if $a$ has finite order then $\ord(a) = \ord(f(a))$.
			\end{enumerate}
			
			\begin{proof} Let $G,K,f$ be as above.
			$\Rightarrow)$ Suppose $G$ is Abelian, we'll show $K$ is Abelian as well, i.e., that for $a,b \in G$ and the values to which they are mapped $f(a),f(b) \in K$ we have $f(a)f(b) = f(b)f(a)$.
			Since $G$ is Abelian we know that for $a,b \in G$ we have $ab = ba$, and since $f$ is an isomorphism we have $$f(a)f(b) = f(ab) = f(ba) = f(b)f(a),$$ as we aimed to do.

			$\Leftarrow)$ Suppose $K$ is Abelian, we'll show that $G$ is Abelian as well.
			Since $K$ is Abelian we know that for $f(a),f(b) \in K,$ we have $f(a)f(b) = f(b)f(a)$.
			Recall that since $f$ is an isomorphism there exists a unique $f^{-1}$ such that $f^{-1}\left( f(ab) \right) = ab$, thus $$f(a)f(b) = f(ab) = f^{-1}\left( f(ab) \right) = ab = ba = f^{-1}\left( f(ba) \right) = f(ba) = f(b)f(a).$$
			Hence, since $K$ is Abelian and $f$ is an isomorphism, $G$ is also Abelian, as we aimed to show.
			\end{proof}
			% subsubsection problem_1_63 (end)

			\subsection*{Problem 1.64}
			\label{sub:problem_1_64}
			Prove (ii) of Theorem 1.40.
			\begin{proof} Let $G,K,f$ be as in the previous problem.
			Suppose $a \in G$ has finite order, $n = \ord(a)$.
			Then,
			\begin{align*}
				a^{n} &= e_{G}, \\
				f(a^{n}) &= f(e_{G}), \\
				\left( f(a) \right)^{n} &= e_{K}.
			\end{align*}
			Thus, $\ord(a) = \ord\left( f(a) \right)$ as we aimed to show.
			
			\end{proof}
			% subsubsection problem_1_64 (end)

			\subsection*{Problem 1.65}
			\label{sub:problem_1_65}
			Suppose $G$ and $K$ are groups and $f : G \to K$ is a homomorphism. 
			Define the set $f(G) = \{ y \in K : \text{there is some $a \in G$ with $f(a) = y$} \}$.
			Prove that if $f(G)$ is a subgroup of $K$.
			\begin{proof} We proceed directly.
			Notice that $f(G)$ is the empty set if and only if $G$ is the trivial ring, thus for nontrivial $G$, $f(G)$ is nonempty.
			Let $x,y \in f(G)$, then there exist $a,b \in G$ such that $f(a) = x, f(b) = y$.
			We compute the following
				\begin{align*}
				x \star y &= f(a) \star f(b), \\
				&= f(a \star b), \\
				&= f(c), \ \ \ \text{for $c = a \star b, c \in G$.}
				\end{align*}
			Since $c \in G$ we know they $f(c) \in f(G)$, thus $f(G)$ is closed under the operation of $(K, \star)$.
			Since $G,K$ are groups, their elements have inverses, therefore $y^{-1} \in K$ such that $y^{-1} = f(a)$ for some $a \in G$.
			Then $y^{-1} \in f(G)$.
			Thus, $f(G)$ is a subgroup of $K$.
			\end{proof}
			% subsubsection problem_1_65 (end)

			\subsection*{Problem 1.66}
			\label{sub:problem_1_66}
			Suppose that $f : G \to K$ and $g : K \to H$ are homomorphisms of the groups $G$, $K$, and $H$. 
			Prove that the function $g \circ f$ is a homomorphism from $G$ to $H$.
			\begin{proof} Let $f,g,G,K,H$ be as above.
			Suppose $a \in G, b = f(a) \in K, c = g(b) \in H$.
			We compute that,
				\begin{align*}
				(g \circ f)(ab) &= (g \circ f)(a) \circ (g \circ f)(b), \\
				&= g\left( f(a)f(b) \right), \\
				&= g(f(a))g(f(b)), \\
				&= (g \circ f)(a) \cdot (g \circ f)(b).
				\end{align*}
			Thus the composition of $f$ and $g$ is also a homomorphism.				
			\end{proof}
			% subsubsection problem_1_66 (end)
		% section homomorphisms (end)

		\section{Groups of Units}
		\label{sec:groups_of_units}
			\subsection*{Problem 2.1}
			\label{sub:problem_2_1}
			Find the elements of $(U(20), \cdot_{20})$ and show its Cayley table.
			\begin{proof}
			\end{proof}
			% subsubsection problem_2_1 (end)

			\subsection*{Problem 2.2}
			\label{sub:problem_2_2}
			Find the elements of $(U(12), \cdot_{12})$ and show its Cayley table.
			\begin{proof}
			\end{proof}
			% subsubsection problem_2_2 (end)

			\subsection*{Problem 2.3}
			\label{sub:problem_2_3}
			Find the elements of $(U(14), \cdot_{14})$ and show its Cayley table.
			\begin{proof}
			\end{proof}
			% subsubsection problem_2_3 (end)

			\subsection*{Problem 2.5}
			\label{sub:problem_2_5}
			Find the elements of $(U(16), \cdot_{16})$ and show its Cayley table.
			\begin{proof}
			\end{proof}
			% subsubsection problem_2_5 (end)

			\subsection*{Problem 2.8}
			\label{sub:problem_2_8}
			Find the elements of $(U(11), \cdot_{11})$ and show its Cayley table.
			\begin{proof}
			\end{proof}
			% subsubsection problem_2_8 (end)

			\subsection*{Problem 2.11}
			\label{sub:problem_2_11}
			In the group $(U(21), \cdot_{21})$ find the order of each element.
			\begin{proof}
			\end{proof}
			% subsubsection problem_2_11 (end)

			\subsection*{Problem 2.12}
			\label{sub:problem_2_12}
			In the group $(U(20), \cdot_{20})$ find the order of each element.
			\begin{proof}
			\end{proof}
			% subsubsection problem_2_12 (end)

			\subsection*{Problem 2.14}
			\label{sub:problem_2_14}
			In the group $(U(12), \cdot_{12})$ find the order of each element.
			\begin{proof}
			\end{proof}
			% subsubsection problem_2_14 (end)

			\subsection*{Problem 2.16}
			\label{sub:problem_2_16}
			In the group $(U(15), \cdot_{15})$ find the order of each element.
			\begin{proof}
			\end{proof}
			% subsubsection problem_2_16 (end)

			\subsection*{Problem 2.17}
			\label{sub:problem_2_17}
			If we have positive integers $n < m$, is $U(n) \subseteq U(m)$?
			Give a proof or counterexample.
			\begin{proof}
			\end{proof}
			% subsubsection problem_2_17 (end)

			\subsection*{Problem 2.18}
			\label{sub:problem_2_18}
			Determine if the set $\{1,3,5,7,9\}$ is a group under the operation $\cdot_{10}$.
			\begin{proof}
			\end{proof}
			% subsubsection problem_2_18 (end)

			\subsection*{Problem 2.19}
			\label{sub:problem_2_19}
			Use Euler's function to determine the size of the set $U(p)$ for an arbitrary prime $p$.
			\begin{proof}
			\end{proof}
			% subsubsection problem_2_19 (end)

			\subsection*{Problem 2.20}
			\label{sub:problem_2_20}
			Prove or disprove: $U(p) = \Z_{p} - \{0\}$ for $p$ prime.
			\begin{proof}
			\end{proof}
			% subsubsection problem_2_20 (end)

			\subsection*{Problem 2.21}
			\label{sub:problem_2_21}
			Suppose we have $n = p^{2}$ when $p$ is prime. 
			Is it true that $U(n) = \Z_{n} - \{0\}$? 
			Be sure to explain why or why not.
			\begin{proof}
			\end{proof}
			% subsubsection problem_2_21 (end)

			\subsection*{Problem 2.22}
			\label{sub:problem_2_22}
			Using Euler's function, explain how to find the cardinality of $U(pq)$ when $p$ and $q$ are distinct primes.
			\begin{proof}
			\end{proof}
			% subsubsection problem_2_22 (end)

			\subsection*{Problem 2.23}
			\label{sub:problem_2_23}
			Using Euler's function, what is the cardinality of $U(35)$?
			\begin{proof}
			\end{proof}
			% subsubsection problem_2_23 (end)

			\subsection*{Problem 2.24}
			\label{sub:problem_2_24}
			Using Euler's function, what is the cardinality of $U(50)$?
			\begin{proof}
			\end{proof}
			% subsubsection problem_2_24 (end)

			\subsection*{Problem 2.25}
			\label{sub:problem_2_25}
			Using Euler's function, what is the cardinality of $U(105)$?
			\begin{proof}
			\end{proof}
			% subsubsection problem_2_25 (end)
		% section groups_of_units (end)

		\section{Cyclic Groups}
		\label{sec:cyclic_groups}
			\subsection*{Problem 2.26}
			\label{sub:problem_2_26}
			Find all of the cyclic subgroups in $(U(20), \cdot_{20})$.
			Is $U(20)$ a cyclic group?
			\begin{proof}
			\end{proof}
			% subsubsection problem_2_26 (end)

			\subsection*{Problem 2.27}
			\label{sub:problem_2_27}
			Find all of the cyclic subgroups in $(U(15), \cdot_{15})$.
			Is $U(15)$ a cyclic group?
			\begin{proof}
			\end{proof}
			% subsubsection problem_2_27 (end)

			\subsection*{Problem 2.28}
			\label{sub:problem_2_28}
			Find all of the cyclic subgroups in $(U(7), \cdot_{7})$.
			Is $U(7)$ a cyclic group?
			\begin{proof}
			\end{proof}
			% subsubsection problem_2_28 (end)

			\subsection*{Problem 2.29}
			\label{sub:problem_2_29}
			Find all of the cyclic subgroups in $(U(12), \cdot_{12})$.
			Is $U(12)$ a cyclic group?
			\begin{proof}
			\end{proof}
			% subsubsection problem_2_29 (end)

			\subsection*{Problem 2.30}
			\label{sub:problem_2_30}
			Find all of the cyclic subgroups in $(U(30), \cdot_{30})$.
			Is $U(30)$ a cyclic group?
			\begin{proof}
			\end{proof}
			% subsubsection problem_2_30 (end)

			\subsection*{Problem 2.31}
			\label{sub:problem_2_31}
			Find all of the cyclic subgroups in $(U(10), \cdot_{10})$.
			Is $U(lO)$ a cyclic group?
			\begin{proof}
			\end{proof}
			% subsubsection problem_2_31 (end)

			\subsection*{Problem 2.32}
			\label{sub:problem_2_32}
			Find all of the cyclic subgroups in $(U(9), \cdot_{9})$.
			Is $U(9)$ a cyclic group?
			\begin{proof}
			\end{proof}
			% subsubsection problem_2_32 (end)

			\subsection*{Problem 2.33}
			\label{sub:problem_2_33}
			Let $G$ denote an arbitrary group with $a,b \in G$. 
			Prove: If $a = b^{k}$ for some integer $k$, $and \ord(a) = \ord(b) = n$ for some $n \in \N$, then $\pid{a} = \pid{b}$.
			\begin{proof}
			\end{proof}
			% subsubsection problem_2_33 (end)

			\subsection*{Problem 2.34}
			\label{sub:problem_2_34}
			We needed \textbf{finite order} elements in the previous exercise since infinite order elements can make it fail! 
			In the group $\Z$ under usual addition, find nonzero integers $a$ and $b$ with $a = b^{k}$ for some nonzero integer $k$, both $a$ and $b$ of infinite order, but $\pid{a} \neq \pid{b}$.
			\begin{proof}
			\end{proof}
			% subsubsection problem_2_34 (end)

			\subsection*{Problem 2.35}
			\label{sub:problem_2_35}
			Find two integers $1 \leq a \leq 5, 1 \leq b \leq 5$ for which $\pid{a} = \pid{b}$ is true in one of the groups $(U(7),\cdot_{7})$ or $(\Z_{6},+_{6})$ but is false in the other group.
			\begin{proof}
			\end{proof}
			% subsubsection problem_2_35 (end)

			\subsection*{Problem 2.36}
			\label{sub:problem_2_36}
			Prove: If $G$ is a cyclic group then $G$ is abelian.
			\begin{proof}
			\end{proof}
			% subsubsection problem_2_36 (end)

			\subsection*{Problem 2.37}
			\label{sub:problem_2_37}
			Let $G$ and $K$ be groups and $f : G \to K$ an isomorphism. 
			Prove: $G$ is cyclic if and only if $K$ is cyclic.
			\begin{proof}
			\end{proof}
			% subsubsection problem_2_37 (end)

			\subsection*{Problem 2.38}
			\label{sub:problem_2_38}
			Let $G$ and $K$ be groups and $f : G \to K$ a homomorphism. 
			Prove or disprove: If $G$ is cyclic then $K$ is cyclic.
			\begin{proof}
			\end{proof}
			% subsubsection problem_2_38 (end)

			\subsection*{Problem 2.39}
			\label{sub:problem_2_39}
			Let $G$ and $K$ be groups. 
			Prove: If $G \times K$ is cyclic then both $G$ and $K$ must be cyclic.
			\begin{proof}
			\end{proof}
			% subsubsection problem_2_39 (end)

			\subsection*{Problem 2.40}
			\label{sub:problem_2_40}
			Prove (ii) of Theorem 2.16.
			\begin{proof}
			\end{proof}
			% subsubsection problem_2_40 (end)

			\subsection*{Problem 2.41}
			\label{sub:problem_2_41}
			Prove (i) of Theorem 2.18.
			\begin{proof}
			\end{proof}
			% subsubsection problem_2_41 (end)

			\subsection*{Problem 2.42}
			\label{sub:problem_2_42}
			Consider the group of functions $\func{\R}$ under addition of functions. 
			Determine the elements of the cyclic subgroup $\pid{f}$ where $f(x) = x - 1$.
			\begin{proof}
			\end{proof}
			% subsubsection problem_2_42 (end)		
		% section cyclic_groups (end)

		\section{Permutation Groups}
		\label{sec:permutation_groups}
			\subsection*{Problem 2.43}
			\label{sub:problem_2_43}
			Prove: If $f:A \to B, g:B \to C$, and $h:C \to D$ are functions, then $h \circ (g \circ f) = (h \circ g) \circ f$.
			\begin{proof}
			\end{proof}
			% subsubsection problem_2_43 (end)

			\subsection*{Problem 2.44}
			\label{sub:problem_2_44}
			Let $A = \{ x \in R : x \neq 0, 1 \}$.
			Determine if $f:A \to A$ defined by $f(x) = 1 - \frac{1}{x}$ is a permutation of $A$.
			\begin{proof}
			\end{proof}
			% subsubsection problem_2_44 (end)

			\subsection*{Problem 2.45}
			\label{sub:problem_2_45}
			Let $A = \{ x \in \R : x \neq 0,1 \}$. 
			Determine if $f:A \to A$ defined by $f(x) = \frac{x}{x-1}$ is a permutation of $A$. 
			\begin{proof}
			\end{proof}
			% subsubsection problem_2_45 (end)

			\subsection*{Problem 2.46}
			\label{sub:problem_2_46}
			Determine if $f: \{ 0,1,2,3,4 \} \to \{ 0,1,2,3,4 \}$ defined by $f(x) = x^{3} \mod{5}$, where $x^{3}$ denotes usual integer multiplication, is a permutation of $\{ 0,1,2,3,4 \}$.
			\begin{proof}
			\end{proof}
			% subsubsection problem_2_46 (end)

			\subsection*{Problem 2.47}
			\label{sub:problem_2_47}
			Determine if $g: \{ 0,1,2,3,4,5 \} \to \{ 0,1,2,3,4,5 \}$ defined by $g(x) = (x+2) \mod{6}$, where $x + 2$ denotes usual integer addition, is a permutation of $\{ 0,1,2,3,4,5 \}$.
			\begin{proof}
			\end{proof}
			% subsubsection problem_2_47 (end)

			\subsection*{Problem 2.48}
			\label{sub:problem_2_48}
			Determine if $h: \{ 0,1,2,3,4,5,6 \} \to \{ 0,1,2,3,4,5,6 \}$ defined by $h(x) = x^{2} \mod{7}$, where $x^{2}$ denotes usual integer multiplication, is a permutation of $\{ 0,1,2,3,4,5,6 \}$.
			\begin{proof}
			\end{proof}
			% subsubsection problem_2_48 (end)

			\subsection*{Problem 2.49}
			\label{sub:problem_2_49}
			Write $\alpha \circ \alpha$, $\beta \circ \beta$, $\alpha \circ \beta$, and $\beta \circ \alpha$ as products of disjoint cycles for the $\alpha,\beta \in S_{7}$ defined in Example 2.33.
			\begin{proof}
			\end{proof}
			% subsubsection problem_2_49 (end)

			\subsection*{Problem 2.50}
			\label{sub:problem_2_50}
			For the permutations $\alpha = (127)(38)(456)$ and $\beta = (23568)(147)$ in $S_{8}$, write $\alpha \circ \alpha$, $\beta \circ \beta$, $\alpha \circ \beta$, and $\beta \circ \alpha$ as products of disjoint cycles.
			\begin{proof}
			\end{proof}
			% subsubsection problem_2_50 (end)

			\subsection*{Problem 2.51}
			\label{sub:problem_2_51}
			For the permutations $\alpha = (1254)(36)$ and $\beta = (14678)$ in $S_{8}$, write $\alpha \circ \alpha$, $\beta \circ \beta$, $\alpha \circ \beta$, and $\beta \circ \alpha$ as products of disjoint cycles.
			\begin{proof}
			\end{proof}
			% subsubsection problem_2_51 (end)

			\subsection*{Problem 2.52}
			\label{sub:problem_2_52}
			For the permutations $\alpha = (234)(567)$ and $\beta = (2358)(147)$ in $S_{8}$, write $\alpha^{2}$, $\alpha^{-1}$, $\alpha \circ \beta$, and $\beta \circ \alpha^{-1}$ as products of disjoint cycles.
			\begin{proof}
			\end{proof}
			% subsubsection problem_2_52 (end)

			\subsection*{Problem 2.53}
			\label{sub:problem_2_53}
			For the permutations $\alpha = (1356)$ and $\beta = (124)(35)$ in $S_{6}$, write $\alpha^{2}$, $\beta^{-1}$, $\alpha \circ \beta$, and $\beta^{-1} \circ \alpha$ as products of disjoint cycles.
			\begin{proof}
			\end{proof}
			% subsubsection problem_2_53 (end)

			\subsection*{Problem 2.54}
			\label{sub:problem_2_54}
			For the permutations $\alpha = (15)(278)(34)$ and $\beta = (12)(368)$ in $S_{8}$, write $\alpha^{2} \circ \beta$, $\beta^{-1} \circ \alpha$, $\alpha^{-2} \circ \beta$, and $\beta^{-3}$ as products of disjoint cycles.
			\begin{proof}
			\end{proof}
			% subsubsection problem_2_54 (end)

			\subsection*{Problem 2.55}
			\label{sub:problem_2_55}
			For the permutations $\alpha = (468)$ and $\beta = (1234)(58)$ in $S_{8}$, write $\alpha^{2} \circ \beta$, $\beta^{-1} \circ \alpha$, $\alpha^{-2} \circ \beta$, and $\beta^{-3}$ as products of disjoint cycles. 
			\begin{proof}
			\end{proof}
			% subsubsection problem_2_55 (end)

			\subsection*{Problem 2.56}
			\label{sub:problem_2_56}
			For the permutations $\alpha = (15)(278)(34)$ and $\beta = (12)(368)$ in $S_{8}$, write $\alpha^{3}$, $\beta \circ \alpha^{2}$, $\alpha^{-1} \circ \beta^{2}$, and $\beta \circ \alpha$ products of disjoint cycles.
			\begin{proof}
			\end{proof}
			% subsubsection problem_2_56 (end)

			\subsection*{Problem 2.57}
			\label{sub:problem_2_57}
			Show that in $S_{k} (k < 1)$ two disjoint cycles will commute. 
			That is, if $\alpha = (a_{1}a_{2} \cdots a_{n})$ and $\beta = (b_{1}b_{2} \cdots b_{m})$ are cycles with $\{ a_{1}, a_{2}, \dots, a_{n} \} \cap \{ b_{1}, b_{2}, \dots, b_{m} \} \neq \emptyset$, then $a \circ b = b \circ a$.
			\begin{proof}
			\end{proof}
			% subsubsection problem_2_57 (end)

			In exercises 58-64, write $\omega$ as a product of transpositions to determine if $\omega$ is even or odd.

			\subsection*{Problem 2.58}
			\label{sub:problem_2_58}
			$\omega = (1234)(5678) \in S_{8}$.
			\begin{proof}
			\end{proof}
			% subsubsection problem_2_58 (end)

			\subsection*{Problem 2.59}
			\label{sub:problem_2_59}
			$\omega = (23456)(17) \in S_{8}$.
			\begin{proof}
			\end{proof}
			% subsubsection problem_2_59 (end)

			\subsection*{Problem 2.60}
			\label{sub:problem_2_60}
			$\omega = (135)(2678) \in S_{8}$.
			\begin{proof}
			\end{proof}
			% subsubsection problem_2_60 (end)

			\subsection*{Problem 2.61}
			\label{sub:problem_2_61}
			$\omega = (123)(456)(789) \in S_{9}$.
			\begin{proof}
			\end{proof}
			% subsubsection problem_2_61 (end)

			\subsection*{Problem 2.62}
			\label{sub:problem_2_62}
			$\omega = (2345678) \in S_{8}$.
			\begin{proof}
			\end{proof}
			% subsubsection problem_2_62 (end)

			\subsection*{Problem 2.63}
			\label{sub:problem_2_63}
			$\omega = (13)(2468) \in S_{8}$.
			\begin{proof}
			\end{proof}
			% subsubsection problem_2_63 (end)

			\subsection*{Problem 2.64}
			\label{sub:problem_2_64}
			$\omega = (14)(356) \in S_{7}$.
			\begin{proof}
			\end{proof}
			% subsubsection problem_2_64 (end)

			\subsection*{Problem 2.65}
			\label{sub:problem_2_65}
			Find an odd permutation of $S_{8}$ which (when written as a product of disjoint cycles) includes a cycle of length 5. 
			What is its order?
			\begin{proof}
			\end{proof}
			% subsubsection problem_2_65 (end)

			\subsection*{Problem 2.66}
			\label{sub:problem_2_66}
			Write the permutation $\omega = (1234)(678) \in S_{8}$ as a product of transpositions two \underline{different ways}. 
			Be sure to check that the product of the transpositions really does give you the same permutation.
			\begin{proof}
			\end{proof}
			% subsubsection problem_2_66 (end)

			\subsection*{Problem 2.67}
			\label{sub:problem_2_67}
			Write the permutation $\omega = (127)(348)(56) \in S_{8}$ as a product of transpositions two \underline{different ways}. 
			Be sure to check that the product of the transpositions really does give you the same permutation.
			\begin{proof}
			\end{proof}
			% subsubsection problem_2_67 (end)

			\subsection*{Problem 2.68}
			\label{sub:problem_2_68}
			Write the permutation $\omega = (13)(456)(27) \in S_{8}$ as a product of transpositions two \underline{different ways}. 
			Be sure to check that the product of the transpositions really does give you the same permutation.
			\begin{proof}
			\end{proof}
			% subsubsection problem_2_68 (end)

			\subsection*{Problem 2.69}
			\label{sub:problem_2_69}
			Prove: For any $k > 1$, a cycle in $S_{k}$ of length $m > 1$ is even (as a permutation) if and only if $m$ is odd (as an integer).
			\begin{proof}
			\end{proof}
			% subsubsection problem_2_69 (end)

			\subsection*{Problem 2.70}
			\label{sub:problem_2_70}
			Prove: For any $k > 1$ and any permutation $\alpha \in S_{k}$, $\alpha^{2}$ must be an even permutation.
			\begin{proof}
			\end{proof}
			% subsubsection problem_2_70 (end)

			\subsection*{Problem 2.71}
			\label{sub:problem_2_71}
			Prove: If $k \geq 3$ and $\alpha \in S_{k}$ is a cycle of length 3, $\alpha = (a_{1}a_{2}a_{3})$, then $\ord(a) = 3$.
			\begin{proof}
			\end{proof}
			% subsubsection problem_2_71 (end)

			\subsection*{Problem 2.72}
			\label{sub:problem_2_72}
			Let $k > 1$ and $a \in S_{k}$ a cycle of length $m > 1$.
			Prove: $a^{2}$ is a cycle if and only if $m$ is odd.
			\begin{proof}
			\end{proof}
			% subsubsection problem_2_72 (end)			
		% section permutation_groups (end)

		\section{$^{\star}$Dihedral Groups}
		\label{sec:dihedral_groups}
		% section dihedral_groups (end)

		\section{$^{\star}$Symmetric Groups}
		\label{sec:_star_symmetric_groups}
		% section _star_symmetric_groups (end)

		\section{$^{\star}$Matrix Groups}
		\label{sec:_star_matrix_groups}
		% section _star_matrix_groups (end)

		\section{$^{\star}$The Quaternion Group}
		\label{sec:_star_the_quaternion_group}
		% section _star_the_quaternion_group (end)		
	% chapter groups (end)

	\chapter{Quotient Groups}
	\label{sec:quotient_groups}
		\subsection{Cosets}
		\label{sub:cosets}
		% subsection cosets (end)

		\subsection{Normal Subgroups}
		\label{sub:normal_subgroups}
		% subsection normal_subgroups (end)

		\subsection{Quotient Groups}
		\label{sub:quotient_groups}
		% subsection quotient_groups (end)

		\subsection{$^{\star}$The Isomorphism Theorems}
		\label{sub:_star_the_isomorphism_theorems}
		% subsection _star_the_isomorphism_theorems (end)

		\subsection{Homomorphic Images of a Group}
		\label{sub:homomorphic_images_of_a_group}
		% subsection homomorphic_images_of_a_group (end)

		\subsection{Theorems of Cauchy and Sylow}
		\label{sub:theorems_of_cauchy_and_sylow}
		% subsection theorems_of_cauchy_and_sylow (end)
	% chapter quotient_groups (end)

	\chapter{$^{\star}$Group Actions}
	\label{sec:group_actions}
		\subsection{$^{\star}$Introduction to Group Actions}
		\label{sub:introduction_to_group_actions}
		% subsection introduction_to_group_actions (end)

		\subsection{$^{\star}$Group Actions and Permutation Representations}
		\label{sub:group_actions_and_permutation_representations}
		% subsection group_actions_and_permutation_representations (end)

		\subsection{$^{\star}$Groups Acting on Themselves by Left Multiplication -- Cayley's Theorem}
		\label{sub:groups_acting_on_themselves_by_left_multiplication_cayley_s_theorem}
		% subsection groups_acting_on_themselves_by_left_multiplication_cayley_s_theorem (end)

		\subsection{$^{\star}$Groups Acting on Themselves by Conjugation -- The Class Equation}
		\label{sub:groups_acting_on_themselves_by_conjugation_the_class_equation}
		% subsection groups_acting_on_themselves_by_conjugation_the_class_equation (end)

		\subsection{$^{\star}$Automorphisms}
		\label{sub:automorphisms}
		% subsection automorphisms (end)

		\subsection{$^{\star}$The Sylow Theorems}
		\label{sub:the_sylow_theorems}
		% subsection the_sylow_theorems (end)

		\subsection{$^{\star}$The Simplicity of $A_{n}$}
		\label{sub:the_simplicity_of_An}
		% subsection the_simplicity_of_An (end)
	% chapter group_actions (end)

	\appendix

	\chapter{Figures}
	\label{sec:figures}
		\subsection{Cayley Tables}
		\label{sub:cayley_tables}
		These are the Cayley tables required for several of the problems involving showing a set and binary operation form a group or subgroup.
		\begin{figure}[h]
			\begin{tabular}{c|cccccc}
				$+_{6}$ & 0 & 1 & 2 & 3 & 4 & 5 \\
				\hline
				0 & 0 & 1 & 2 & 3 & 4 & 5 \\
				1 & 1 & 2 & 3 & 4 & 5 & \color{red} 0 \color{black} \\
				2 & 2 & 3 & 4 & 5 & \color{red} 0 \color{black} & 1 \\
				3 & 3 & 4 & 5 & \color{red} 0 \color{black} & 1 & 2 \\
				4 & 4 & 5 & \color{red} 0 \color{black} & 1 & 2 & 3 \\
				5 & 5 & \color{red} 0 \color{black} & 1 & 2 & 3 & 4
			\end{tabular}
			\caption{Cayley Table for $(\Z_{6}, +_{6})$ from Problem 1.3.17}
			\label{117CT}
		\end{figure}

		\begin{figure}[h]
			\begin{tabular}{c|cccccccccc}
				$*_{5 \times 2}$ & (0,0) & (0,1) & (1,0) &(1,1) & (2,0) & (2,1) & (3,0) & (3,1) & (4,0) & (4,1) \\
				\hline
				(0,0) & \color{red}(0,0)\color{black} & (0,1) & (1,0) & (1,1) & (2,0) & (2,1) & (3,0) & (3,1) & (4,0) & (4,1) \\
				(0,1) & (0,1) & \color{red}(0,0)\color{black} & (1,1) & (1,0) & (2,1) & (2,0) & (3,1) & (3,0) & (4,1) & (4,0) \\
				(1,0) & (1,0) & (1,1) & (2,0) & (2,1) & (3,0) & (3,1) & (4,0) & (4,1) & \color{red}(0,0)\color{black} & (0,1) \\
				(1,1) & (1,1) & (1,0) & (2,1) & (2,0) & (3,1) & (3,0) & (4,1) & (4,0) & (0,1) & \color{red}(0,0)\color{black} \\
				(2,0) & (2,0) & (2,1) & (3,0) & (3,1) & (4,0) & (4,1) & \color{red}(0,0)\color{black} & (0,1) & (1,0) & (0,1) \\
				(2,1) & (2,1) & (2,0) & (3,1) & (3,0) & (4,1) & (4,0) & (0,1) & \color{red}(0,0)\color{black} & (1,1) & (1,0) \\
				(3,0) & (3,0) & (3,1) & (4,0) & (4,1) & \color{red}(0,0)\color{black} & (0,1) & (1,0) & (1,1) & (2,0) & (2,1) \\
				(3,1) & (3,1) & (3,0) & (4,1) & (4,0) & (0,1) & \color{red}(0,0)\color{black} & (1,1) & (1,0) & (2,1) & (2,0) \\
				(4,0) & (4,0) & (4,1) & \color{red}(0,0)\color{black} & (0,1) & (1,0) & (1,1) & (2,0) & (2,1) & (3,0) & (3,1) \\
				(4,1) & (4,1) & (4,0) & (0,1) & \color{red}(0,0)\color{black} & (1,1) & (1,0) & (2,1) & (2,0) & (3,1) & (3,0)  
			\end{tabular}
			\caption{Cayley Table for $(\Z_{5} \times \Z_{2}, *_{5 \times 2})$ from Problem 1.3.18}
			\label{118CT}
		\end{figure}

		\begin{figure}[h]
			\begin{tabular}{c|cccccc}
				$*$ & S & H & V & D  \\
				\hline
				S & \color{red}S\color{black} & H & V & D \\
				H & H & \color{red}S\color{black} & D & V \\
				V & V & D & \color{red}S\color{black} & H \\
				D & D & V & H & \color{red}S\color{black} \\
			\end{tabular}
			\caption{Cayley Table for $(G, *)$ from Problem 1.21}
			\label{121CT}
		\end{figure}

		\begin{figure}[t]
			\begin{tabular}{c|cccc}
				$\times$ & $\begin{pmatrix} 1 & 0 \\ 0 & -1 \end{pmatrix}$ & $\begin{pmatrix} 1 & 0 \\ 0 & 1 \end{pmatrix}$ & $\begin{pmatrix} -1 & 0 \\ 0 & 1 \end{pmatrix}$ & $\begin{pmatrix} -1 & 0 \\ 0 & -1 \end{pmatrix}$ \\
				\hline
				$\begin{pmatrix} 1 & 0 \\ 0 & -1 \end{pmatrix}$ & \color{red}$\begin{pmatrix} 1 & 0 \\ 0 & 1 \end{pmatrix}$\color{black} & $\begin{pmatrix} 1 & 0 \\ 0 & -1 \end{pmatrix}$ & $\begin{pmatrix} -1 & 0 \\ 0 & -1 \end{pmatrix}$ & $\begin{pmatrix} -1 & 0 \\ 0 & 1 \end{pmatrix}$ \\
				$\begin{pmatrix} 1 & 0 \\ 0 & 1 \end{pmatrix}$ & $\begin{pmatrix} 1 & 0 \\ 0 & -1 \end{pmatrix}$ & \color{red}$\begin{pmatrix} 1 & 0 \\ 0 & 1 \end{pmatrix}$\color{black} & $\begin{pmatrix} -1 & 0 \\ 0 & 1 \end{pmatrix}$ & $\begin{pmatrix} -1 & 0 \\ 0 & -1 \end{pmatrix}$ \\
				$\begin{pmatrix} -1 & 0 \\ 0 & 1 \end{pmatrix}$ & $\begin{pmatrix} -1 & 0 \\ 0 & -1 \end{pmatrix}$ & $\begin{pmatrix} -1 & 0 \\ 0 & 1 \end{pmatrix}$ & \color{red}$\begin{pmatrix} 1 & 0 \\ 0 & 1 \end{pmatrix}$\color{black} & $\begin{pmatrix} 1 & 0 \\ 0 & -1 \end{pmatrix}$ \\
				$\begin{pmatrix} -1 & 0 \\ 0 & -1 \end{pmatrix}$ & $\begin{pmatrix} -1 & 0 \\ 0 & 1 \end{pmatrix}$ & $\begin{pmatrix} -1 & 0 \\ 0 & -1 \end{pmatrix}$ & $\begin{pmatrix} 1 & 0 \\ 0 & -1 \end{pmatrix}$ & \color{red}$\begin{pmatrix} 1 & 0 \\ 0 & 1 \end{pmatrix}$ \color{black}
			\end{tabular}
			\caption{Cayley Table for $(A, \times)$ from Problem 1.22}
			\label{122CT}
		\end{figure}

		\begin{figure}[h]
			\begin{tabular}{c|cccc}
				$+_{12}$ & 0 & 3 & 6 & 9 \\
				\hline
				0 & \color{red} 0 \color{black} & 3 & 6 & 9 \\
				3 & 3 & 6 & 9 & \color{red} 0 \color{black} \\
				6 & 6 & 9 & \color{red} 0 \color{black} & 3 \\
				9 & 9 & \color{red} 0 \color{black} & 3 & 6
			\end{tabular}
			\caption{Cayley Table for Problem 1.40}
			\label{140CT}
		\end{figure}

		\begin{figure}[h]
			\begin{tabular}{c|ccccc}
				$+_{18}$ & 0 & 2 & 4 & 8 & 16 \\
				\hline
				0  & \color{red} 0 \color{black}  & 2  & 4  & 8  & 16 \\
				2  & 2  & 4  & 6  & 10 & \color{red} 0 \color{black} \\
				4  & 4  & 6  & 8  & 12 & 2 \\
				8  & 8  & 10 & 12 & 16 & 6 \\
				16 & 16 & \color{red} 0 \color{black}  & 2  & 6  & 4
			\end{tabular}
			\caption{Cayley Table for Problem 1.42}
			\label{142CT}
		\end{figure}

		\begin{figure}[h]
			\begin{tabular}{c|cccc}
				$*$ & (0,0) & (0,2) & (1,0) & (1,2) \\
				\hline
				(0,0) & \color{red} (0,0) \color{black} & (0,2) & (1,0) & (1,2) \\
				(0,2) & (0,2) & \color{red} (0,0) \color{black} & (1,2) & (1,0) \\
				(1,0) & (1,0) & (1,2) & \color{red} (0,0) \color{black} & (0,2) \\
				(1,2) & (1,2) & (1,0) & (0,2) & \color{red} (0,0) \color{black}
			\end{tabular}
			\caption{Cayley Table for Problem 1.43}
			\label{143CT}
		\end{figure}

		\begin{figure}[h]
			\begin{tabular}{c|cc}
				$+_{4}$ & $f(a) = 0$ & $f(b) = 2$ \\
				\hline
				$f(a) = 0$ & 0 & 2 \\
				$f(b) = 2$ & 2 & 0
			\end{tabular}
			\caption{Cayley Table for part of Problem 1.54 where $f(a),f(b) \in \Z_{4}$, and $a,b \in \Z_{8}$}
			\label{154CT}
		\end{figure}
		% subsection cayley_tables (end)

		\subsection{Tables for Functions on Finite Sets}
		\label{sub:tables_for_functions_on_finite_sets}
		These are the tables that define some functions over finite sets for certain problems.\\

		\begin{figure}[h]
			\begin{tabular}{c|c}
				$a$ & $2a \mod{4}$ \\
				\hline
				0 & 0 \\
				1 & 2 \\
				2 & 0 \\
				3 & 2 \\
				4 & 0 \\
				5 & 2 \\
				6 & 0 \\
				7 & 2
			\end{tabular}
			\caption{Function table for $f:\Z_{8}\to\Z_{4}$ mapped by $a \mapsto 2a \mod{4}$ from Problem 1.54.}
			\label{154f1}
		\end{figure}
		% subsection tables_for_functions_on_finite_sets (end)
	% chapter figures (end)

	\backmatter
\end{document}